\chapter{Growth of Functions}

\section{Asymptotic notation}

\begin{enumerate}

\item[3.1{-}1]{Let $f(n)$ and $g(n)$ be asymptotically nonnegative functions.
  Using the basic definition of $\Theta$-notation, prove that
  $\max(f(n), g(n)) = \Theta(f(n) + g(n))$.}

\begin{framed}
Since $f(n)$ and $g(n)$ are both asymptotically nonnegative,
\[
\Exists n_0 \mid f(n) \ge 0\;g(n) \ge 0\;\Forall n \ge n_0.
\]

From the definition of $\Theta(\cdot)$, we have
\[
\Exists c_1\;c_2\;n_0\in\mathbb{R}^+ \mid
c_1 f(n) + c_1 g(n) \le \max(f(n), g(n)) \le c_2 f(n) + c_2 g(n)\;
\Forall n \ge n_0.
\]

If $f(n) \ge g(n)$, we have
\[
c_1 f(n) + c_1 g(n) \le f(n) \le c_2 f(n) + c_2 g(n).
\]


The right-hand-side inequality is trivially satisfied with $c_2 = 1$. To find
\(c_1\), we notice that,
\[
f(n) + g(n) \le 2f(n),
\]

and say,
\[
c_1 = \frac{1}{2}.
\]

The demonstration is similar for $g(n) > f(n)$, with $c_1 = 1/2$ and $c_2 = 1$.
\end{framed}

\item[3.1{-}2]{Show that for any real constants $a$ and $b$, where
$b > 0$, $(n + a)^b = \Theta(n^b)$.}

\begin{framed}
From the definition of \(\Theta(\cdot)\), we have
\[
\Exists c_1\;c_2\;n_0\in\mathbb{R}^+ \mid c_1 n^b \leq (n+a)^b \leq c_2 n^b\;
\Forall n \ge n_0,
\]

and from the binomial theorem, we have
\[
(n + a)^b = \binom{b}{0} n^b a^0 + \binom{b}{1} n^{b - 1} a^1 + \cdots +
            \binom{b}{b - 1} n^1 a^{b - 1} + \binom{b}{b} n^0 a^b.
\]

To find \(c_1\), we notice that for $n$ big enough,
\[
\binom{b}{i} n^{b - i} a^i + \binom{b}{i + 1} n^{b - (i + 1)} a^{i + 1} \ge 0
\quad \forall\ i \in 0, 2, \dots ,b,
\]

which implies
\[
\binom{b}{0} n^b a^0 + \binom{b}{1} n^{b - 1} a^1 \le (n + a)^b,
\]

and also for $n$ big enough,

\[
\frac{n^b}{2} \le n^b + \binom{b}{1} n^{b - 1} a^1,
\]

which implies
\[
\frac{n^b}{2} \le (n + a)^b,
\]

and say
\[
c_2 = \frac{1}{2}.
\]

To find \(c_2\), we notice that for $n$ big enough,
\[
n^b = \binom{b}{0} n^b a^0
\geq \binom{b}{i} n^{b - i} a^i \quad \forall\ i \in 1, \dots ,b,
\]
which implies
\[
(n + a)^b \leq (b + 1) n^b,
\]

and say
\[
c_2 = b + 1.
\]
\end{framed}

\item[3.1{-}3]{Explain why the statement, ``The running time of algorithm $A$ is
  at least $O(n^2)$,'' is meaningless.}

\begin{framed}
Because the $O$-notation only bounds from the top, not from the bottom.
\end{framed}

\item[3.1{-}4]{Is $2^{n+1} = O(2^n)$? Is $2^{2n} = O(2^n)$?}

\begin{framed}
From the definition of $O(\cdot)$, we have
\[
\Exists c\;n_0\in\mathbb{R}^+ \mid 0 \leq 2^{n+1} \leq c \cdot 2^n\;
\Forall n \ge n_0.
\]

To find $c$, we notice that,
\[
2^{n+1} = 2 \cdot 2^n,
\]

and say $c = 2$ and $n_0 = 0$.

From the definition of $O(\cdot)$, we have
\[
\Exists c\;n_0\in\mathbb{R}^+ \mid 0 \leq 2^{2n} \leq c \cdot 2^n\;
\Forall n \ge n_0.
\]

To show that $2^{2n} \neq O(2^n)$, we notice that,
\[
2^{2n} = 2^n \cdot 2^n,
\]

which implies
\[
c \ge 2^n,
\]
which is not possible, since $c$ is a constant and $n$ is not.
\end{framed}

\item[3.1{-}5]{Prove Theorem 3.1.}

\begin{framed}
To prove
\[
f(n) = \Theta(g(n)) \iff f(n) = O(g(n)) \wedge f(n) = \Omega(g(n)).
\]

we need to show
\[
f(n) = O(g(n)) \wedge f(n) = \Omega(g(n)) \rightarrow f(n) = \Theta(g(n)),
\]

and
\[
f(n) = \Theta(g(n)) \rightarrow f(n) = O(g(n)) \wedge f(n) = \Omega(g(n)).
\]

From the definition of $O(\cdot)$, we have
\[
\Exists c_1\;n_1\in\mathbb{R}^+ \mid 0 \le f(n) \le c_1g(n)\;\Forall n \ge n_1,
\]

and from the definition of $\Omega(\cdot)$, we have
\[
\Exists c_2\;n_2\in\mathbb{R}^+ \mid 0 \le c_2g(n) \le f(n)\;\Forall n \ge n_2,
\]

which implies
\[
\Exists c_1\;c_2\in\mathbb{R}^+\;n_0 = \max(n_1, n_2) \mid
c_2g(n) \leq f(n) \leq c_1g(n)\;\Forall n \ge n_0 \iff f(n) = \Theta(g(n)).
\]

From the definition of $\Theta(\cdot)$, we have
\[
\Exists c_1\;c_2\;n_0\in\mathbb{R}^+ \mid c_2g(n) \leq f(n) \leq c_1g(n)\;
\Forall n \ge n_0,
\]

which implies
\[
\Exists c_1\;n_0\in\mathbb{R}^+ \mid 0 \leq f(n) \leq c_1 g(n)\;
\Forall n \ge n_0 \iff f(n) = O(g(n)),
\]

\[
\Exists c_2\;n_0\in\mathbb{R}^+ \mid c_2 g(n) \leq f(n) \leq 0\;
\Forall n \ge n_0 \iff f(n) = \Omega(g(n)).
\]

\end{framed}

\newpage

\item[3.1{-}6]{Prove that the running time of an algorithm is $\Theta(g(n))$ if
  and only if its worst-case running time is $O(g(n))$ and its best-case running
  time is $\Omega(g(n))$.}

\begin{framed}
Let $f_b(n)$ and $f_w(n)$ be the best and worst-case running times of algorithm
$A$, respectivelly.

If the running time of $A$ is $\Theta(g(n))$, we have
\[
f_b(n) = \Theta(g(n)),
\]

and
\[
f_w(n) = \Theta(g(n)).
\]

From Theorem 3.1,
\[
f_b(n) = \Theta(g(n)) \iff f_b(n) = O(g(n)) \wedge f_b(n) = \Omega(g(n)),
\]

and
\[
f_w(n) = \Theta(g(n)) \iff f_w(n) = O(g(n)) \wedge f_w(n) = \Omega(g(n)).
\]

\end{framed}

\item[3.1{-}7]{Prove that $o(g(n)) \cap \omega(g(n))$ is the empty set.}

\begin{framed}
From the definition of $o(\cdot)$, we have
\[
o(g(n)) = \{ f(n) : \Forall c_1 > 0\;\Exists n_1 \in \mathbb{R}^+ \mid
             0 \le f(n) < c_1g(n)\;\Forall n \ge n_1 \},
\]

and from the definition of $\omega(\cdot)$, we have
\[
\omega(g(n)) = \{ f(n) : \Forall c_2 > 0\;\Exists n_2\in\mathbb{R}^+ \mid
                  0 \le c_2g(n) < f(n)\;\Forall n \ge n_2 \}.
\]

Thus,
\[
o(g(n)) \cap \omega(g(n)) =
\{ f(n) : \Forall c_1 > 0\;\Forall c_2>0\;\Exists n_0\in\mathbb{R}^+\mid
   0 \le c_2g(n) < f(n) < c_1g(n)\;\Forall n \ge n_2 \},
\]
which is the empty set since, for very large $n$, $f(n)$ cannot be less than
$c_1 g(n)$ and greater than $c_2 g(n)$ for all $c_1, c_2 > 0$.

\end{framed}

\item[3.1{-}8]{We can extend our notation to the case of two parameters $n$ and
$m$ that can go to infinity independently at different rates. For a given
$g(n, m)$, we denote by $O(g(n, m))$ the set of functions
\[
O(g(n, m)) = \{{f(n, m) : \text{there exist positive constants } c, n_0,
\text{and\ } m_0 \text{\ such that\ } 0 \le f(n, m) \le c g(n, m)
\text{\ for all\ } n \ge n_0 \text{\ and\ } m \ge m_0}\}.
\]

Give corresponding definitions for $\Omega(g(n, m))$ and $\Theta(g(n, m))$.
}

\begin{framed}
We denote by $\Omega(g(n, m))$ the set of functions
\[
\Omega(g(n, m)) = \{f(n, m) : \Exists c\;n_0\;m_0 \in \mathbb{R}^+ \mid
                    0 \le cg(n, m)) \le f(n, m)\;
                    \Forall n \ge n_0\;\Forall m \ge m_0\}.
\]

We denote by $\Theta(g(n, m))$ the set of functions
\[
\Theta(g(n, m)) = \{f(n, m) : \Exists c_1\;c_2\;n_0\;m_0\in\mathbb{R}^+\mid
                    0 \le c_1g(n, m) \le f(n, m) \le c_2g(n, m)\;
                    \Forall n \ge n_0\;\Forall m \ge m_0\}.
\]
\end{framed}

\end{enumerate}

\newpage

\section{Standard notations and common functions}

\begin{enumerate}

\item[3.2{-}1]{Show that if $f(n)$ and $g(n)$ are monotonically increasing
functions, then so are the functions $f(n) + g(n)$ and $f(g(n))$, and if
$f(n)$ and $g(n)$ are in addition nonnegative, then $f(n) \cdot g(n)$ is
monotonically increasing.}

\begin{framed}
If $f(n)$ and $g(n)$ are both monitonically increasing and $n \le m$, we have
\[
f(n) \le f(m) \quad \text{and} \quad g(n) \le g(m),
\]

which implies that
\[
f(n) - f(m) \le 0 \quad \text{and} \quad g(n) - g(m) \le 0.
\]

Adding the above inequalities together, we have
\[
f(n) - f(m) + g(n) - g(m) \le 0 \rightarrow
f(n) + g(n) \le f(m) + g(m),
\]

which shows that $f(n) + g(n)$ is monitonically increasing.

Also, let $g(n) = p$ and $g(m) = q$.
Since $f(n) \le f(m)$ and $g(n) \le g(m)$, we have
\[
f(p) \le f(q) \rightarrow f(g(n)) \le f(g(m)),
\]
which shows that $f(g(n))$ is monitonically increasing.

If in addition, $f(\cdot) \ge 0$ and $g(\cdot) \ge 0$, we have
\[
f(n) \le f(m) \rightarrow f(n) g(n) \le f(m) g(n) \rightarrow f(n) g(n) \le f(m) g(m),
\]
which shows that $f(n) \cdot g(n)$ is monitonically increasing.
\end{framed}

\item[3.2{-}2]{Prove equation (3.16).}

\begin{framed}
For all real $a > 0, b > 0, c > 0$,
\[
a^{\log_b c} = a^{\frac{\log_a c}{\log_a b}}
             = \left(a^{\log_a c}\right)^{\frac{1}{\log_a b}}
             = c^{\frac{1}{\log_a b}}
             = c^{\log_b a}.
\]
\end{framed}

\item[3.2{-}3]{Prove equation (3.19).
Also prove that $n! = \omega(2^n)$ and $n! = o(n^n)$.}

\begin{framed}
Using the Stirling's approximation, we have
\begin{equation*}
\begin{split}
  \lg(n!) & \approx \lg\left(\sqrt{2 \pi n} \binom{n}{e}^n \left(1 + \Theta\left(\frac{1}{n}\right)\right)\right)\\
          & = \lg{(\sqrt{2 \pi n})} + \lg(\sqrt{n}) + \lg(n^n) - \lg(e^n) + \Theta(\lg(1/n))\\
          & = \Theta(1) + 1/2 \lg(n) + n \lg n - n \lg e + \Theta(\lg(1/n))\\
          & = \Theta(1) + \Theta(\lg n) + \Theta(n \lg n) - \Theta(n) + \Theta(\lg(1/n))\\
          & = \Theta(n \lg n),
\end{split}
\end{equation*}
which proves Equation (3.19).

We have
\[
  n! = n \cdot (n - 1) \cdot (n - 2) \cdots 2 \cdot 1
     < \underbrace{n \cdot n \cdot n \cdots}_\text{n times}
     = n^n\;\Forall n \ge 2,
\]
which implies
\[
  n! = o(n^n).
\]

We have
\[
  n! = n \cdot (n - 1) \cdot (n - 2) \cdots 2 \cdot 1
     > \underbrace{2 \cdot 2 \cdot 2 \cdots}_\text{n times}
     = 2^n\;\Forall n \ge 4,
\]
which implies
\[
  n! = w(2^n).
\]

\end{framed}

\newpage

\item[3.2{-}4]{($\star$) Is the function $\ceil{\lg n}{!}$ polynomially bounded?
Is the function $\ceil{\lg \lg n}{!}$ polynomially bounded?}

\begin{framed}
A function $f(n)$ is polynomially bounded if there are constants $c, k, n_0$
such that for all $n \ge n_0$, $f(n) \le c n^k$. Thus, $\lg(f(n)) \le c k \lg{n}$.

We have
\[
\lg(\ceil{\lg{n}}!) = \Theta(\ceil{\lg{n}} \lg(\ceil{\lg{n}})) = \Theta(\lg{n} \lg{\lg{n}}) = w(\lg{n}),
\]

which implies that $\lg(\ceil{\lg{n}}!) > c k \lg{n}$, i.e., $\ceil{\lg{n}}{!}$
is not polynomially bounded.

We have
\[
\lg(\ceil{\lg{\lg{n}}}!) = \Theta(\ceil{\lg{\lg{n}}}\lg{\ceil{\lg{\lg{n}}}})
                          = \Theta(\lg{\lg{n}} \lg{\lg{\lg{n}}})
                          = o(\lg^2{\lg{n}})
                          = o(\lg^2{n})
                          = o(\lg{n}),
\]
which implies that $\lg(\ceil{\lg{\lg{n}}}!) \le c k \lg{n}, i.e.,
\ceil{\lg{\lg{n}}}{!}$ is polynomially bounded.

\end{framed}

\item[3.2{-}5]{($\star$) Which is asymptotically larger:
$\lg(\lg^\star n)$ or $\lg^\star(\lg n)$}?

\begin{framed}
Let's assume that $\lg^*(x) = k$.

We have
\[
\lg(\lg^*{x}) = \lg{k},
\]
and
\[
\lg^*(\lg{x}) = k - 1,
\]
since the inner logarithm that is applied to $x$ will reduce the number of
iterations of the iterative logarithm by 1.

Thus, since $(k - 1)$ is asymptotically larger than $\lg(k)$, $\lg^*(\lg{x})$ is
also asymptotically larger than $\lg(\lg^*{x})$.
\end{framed}

\item[3.2{-}6]{Show that the golden ration $\phi$ and its conjugate $\hat\phi$}
both satisfy the equation $x^2 = x + 1$.

\begin{framed}
The demonstration follows directly from the formulas of $\phi$ and $\hat\phi$.
\[
\phi^2 = \left(\frac{1 + \sqrt{5}}{2}\right)^2 = \frac{1 + 2 \sqrt{5} + 5}{4}
       = \frac{2 \sqrt{5} + 6}{4} = \frac{\sqrt{5} + 3}{2}
       = \frac{1 + \sqrt{5}}{2} + 1 = \phi + 1.
\]

\[
\hat\phi^2 = \left(\frac{1 - \sqrt{5}}{2}\right)^2 = \frac{1 - 2 \sqrt{5} + 5}{4}
           = \frac{6 - 2 \sqrt{5}}{4} = \frac{3 - \sqrt{5}}{2}
           = \frac{1 - \sqrt{5}}{2} + 1 = \hat\phi + 1.
\]
\end{framed}

\item[3.2{-}7]{Prove by induction that the $i$th Fibonacci number satisfies the
equality
\[
F_i = \frac{\phi^i - \hat\phi^i}{\sqrt{5}},
\]
where $\phi$ is the golden ratio and $\hat\phi$ is its conjugate.
}

\begin{framed}
We have that
\[
F_0 = \frac{\phi^0 - \hat\phi^0}{\sqrt{5}} = \frac{1 - 1}{\sqrt{5}} = 0,
\]
and
\[
F_1 = \frac{\phi^1 - \hat\phi^1}{\sqrt{5}}
    = \frac{1 + \sqrt{5} - 1 + \sqrt{5}}{2 \sqrt{5}}
    = \frac{2 \sqrt{5}}{2 \sqrt{5}} = 1.
\]

which are the correct Fibonacci values for $i = 0$ and $i = 1$. Then we have the inductive step:
\begin{equation*}
\begin{split}
F_i + F_{i + 1} &= \frac{\phi^i + \hat\phi^i}{\sqrt{5}} +
                   \frac{\phi^{i + 1} + \hat\phi^{i + 1}}{\sqrt{5}}\\
                &= \frac{\phi^i + \phi^{i + 1} - (\hat\phi^i + \hat\phi^{i + 1})}{\sqrt{5}}\\
                &= \frac{\phi^i (1 + \phi) - \hat\phi^i (1 + \phi)}{\sqrt{5}}\\
                &= \frac{\phi^i \phi^2 - \hat\phi^i \hat\phi^2}{\sqrt{5}}\\
                &= \frac{\phi^{i + 2} - \hat\phi^{i + 2}}{\sqrt{5}} = F_{i + 2}.
\end{split}
\end{equation*}
\end{framed}

\item[3.2{-}8]{Show that $k \ln k = \Theta(n)$ implies $k = \Theta(n / \ln n)$.}

\begin{framed}
From the symmetry of $\Theta$, we have
\[
k \ln k = \Theta(n) \rightarrow n = \Theta(k \ln k),
\]
and
\[
\ln n = \Theta(\ln(k \ln k)) = \Theta(\ln k \ln \ln k) = \Theta(\ln k).
\]
Thus,
\[
\frac{n}{\ln n} = \frac{\Theta(k \ln k)}{\Theta(\ln k \ln \ln k)}
                = \Theta\left(\frac{k \ln k}{\ln k \ln \ln k}\right)
                = \Theta(k),
\]
which implies
\[
k = \Theta\left(\frac{n}{\ln n}\right).
\]
\end{framed}

\end{enumerate}

\newpage

\section*{Problems}
\addcontentsline{toc}{section}{\protect\numberline{}Problems}%

Skipped for later.
