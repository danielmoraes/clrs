\documentclass{report}

%                       MAIN PACKAGES                       %
% --------------------------------------------------------- %

\usepackage[english]{babel}
\usepackage{graphicx}
\usepackage{caption}
\usepackage{isolatin1}
\usepackage{amssymb}
\usepackage{textcomp}
\usepackage[usenames,dvipsnames]{color}
\usepackage{soul}
\usepackage{cancel}
\usepackage{booktabs}
\usepackage{multirow}
\usepackage{framed}
\usepackage{xcolor}
\usepackage{algpseudocode}
\usepackage[linesnumbered]{algorithm2e} % for psuedo code
\usepackage{courier}
\usepackage{mathtools} % loads amsmath
\usepackage{forest}
\usepackage{tikz}
\usepackage{tikz-qtree}
\usepackage{interval}
\usetikzlibrary{calc}

%                         MARGINS                           %
% --------------------------------------------------------- %

\hoffset=-0.5in
\voffset=-0.6in
\oddsidemargin=0pt
\topmargin=0pt
\headheight=12pt
\headsep=15pt
\textheight=690pt
\textwidth=543pt
\marginparsep=11pt
\marginparwidth=54pt
\footskip=25pt
\marginparpush=5pt
\paperwidth=597pt
\paperheight=845pt

%                        PAGE STYLE                         %
% --------------------------------------------------------- %

\usepackage{fancyhdr}
\pagestyle{fancy}
\lhead{CLRS {--} Chapter 8 {--} Sorting in Linear Time}
\rhead{Daniel Bastos Moraes}
\cfoot{}
\rfoot{\thepage}
\renewcommand{\headrulewidth}{0.4pt}
\renewcommand{\footrulewidth}{0.4pt}
\newcommand{\Perp}{\perp\! \! \! \perp}

%                       PROPERTIES                          %
% --------------------------------------------------------- %

\setlength{\parindent}{0cm}

\makeatletter
\renewenvironment{framed}{%
 \def\FrameCommand##1{\hskip\@totalleftmargin
 \fboxsep=\FrameSep\fbox{##1}}%
 \MakeFramed {\advance\hsize-\width
   \@totalleftmargin\z@ \linewidth\hsize
   \@setminipage}}%
 {\par\unskip\endMakeFramed}
\makeatother

\makeatletter
\def\BState{\State\hskip-\ALG@thistlm}
\makeatother

\DeclarePairedDelimiter{\ceil}{\lceil}{\rceil}
\DeclarePairedDelimiter{\floor}{\lfloor}{\rfloor}
\DeclareMathOperator{\Exists}{\exists}
\DeclareMathOperator{\Forall}{\forall}

\def\figuredirectory{images}

\mathchardef\mhyphen="2D % Define a "math hyphen"

\tikzset{every tree node/.style={minimum width=1.95em,draw,circle,font=\footnotesize},
         blank/.style={draw=none},
         edge from parent/.style=
         {draw, edge from parent path={(\tikzparentnode) -- (\tikzchildnode)}},
         level distance=1cm}

\DeclareMathOperator{\di}{d\!}
\newcommand*\Eval[3]{\left.#1\right\rvert_{#2}^{#3}}

\intervalconfig{soft open fences}

\let\oldnl\nl% Store \nl in \oldnl
\newcommand{\nonl}{\renewcommand{\nl}{\let\nl\oldnl}}% Remove line number for one line

%                        DOCUMENT                           %
% --------------------------------------------------------- %

\begin{document}

\small

{\large Section 8.1 {--} Lower bounds for sorting}

\begin{enumerate}

\item[8.1{-}1]{What is the smallest possible depth of a leaf in a decision tree
for a comparison sort?}

\begin{framed}
The smallest possible depth of a leaf in a decision tree
can be obtained by calculating the shortest simple path from the root to any
of its reachable leaves. This smallest path occurs when the comparisons is made
in the sorted order. For instance, if the input array is sorted, the following
comparisons suffices
\begin{equation*}
\begin{aligned}
  a_1 &\le a_2,\\
  a_2 &\le a_3,\\
  &\;\;\vdots\\
  a_{n - 1} &\le a{n}.
\end{aligned}
\end{equation*}

Thus, the smallest depth of a leaf is any decision tree is $n - 1 = \Theta(n)$.

\end{framed}

\item[8.1{-}2]{Obtain asymptotically tight bounds on $\lg(n!)$ without using
Stirling's approximation. Instead, evaluate the summation $\sum_{k = 1}^n \lg k$
using techniques from Section A.2.}

\begin{framed}
Assume for convenience that $n$ is even. For a lower bound, we have
\begin{equation*}
\begin{aligned}
  \lg{n!} &=   \lg(n \cdot (n - 1) \cdot (n - 2) \cdots 1)\\
          &=   \sum_{k = 1}^{n} \lg k\\
          &=   \sum_{k = 1}^{n/2} \lg k + \sum_{k = n/2 + 1}^{n} \lg k\\
          &\ge \sum_{k = 1}^{n/2} 0 + \sum_{k = n/2 + 1}^{n} \lg (n/2)\\
          &=   \frac{n}{2} \lg{\frac{n}{2}}\\
          &=   \frac{n}{2} \lg n - \frac{n}{2}\\
          &=   \Omega(n \lg n).
\end{aligned}
\end{equation*}

And for an upper bound, we have
\begin{equation*}
\begin{aligned}
  \lg{n!} &=   \lg(n \cdot (n - 1) \cdot (n - 2) \cdots 1)\\
          &=   \sum_{k = 1}^{n} \lg k\\
          &\le \sum_{k = 1}^{n} \lg n\\
          &=   O(n \lg n).
\end{aligned}
\end{equation*}

Thus, $\lg n! = \Theta(n \lg n)$.
\end{framed}

\newpage

\item[8.1{-}3]{Show that there is no comparison sort whose running time is
linear for at least half of the $n!$ inputs of length $n$. What about a fraction
of $1/n$ of the inputs of length $n$?  What about a fraction $1/2^n$?}

\begin{framed}
Such algorithm only exists if we can build a decision tree such that at least
$n!/2$ of its $n!$ leaves has a depth of $\Theta(n)$. Suppose this decision tree
exists. Let $m$ be the depth of the leaf with the ($n!/2$)th smallest depth.
Remove all nodes with depth greater than $m$. The result is a decision tree
with height $m$ and at least $n!/2$ leaves. Using the same reasoning as in the
proof of Theorem 8.1, for every decision tree with at least $n!/2$ leaves, we
have
\[
  \frac{n!}{2} \le l \le 2^{m},
\]
which implies
\[
  m \ge \lg \frac{n!}{2} = \lg n! - 1 = \Omega(n \lg n),
\]
which proves that such a decision tree does not exists. The same reasoning can
be applied to obtain the maximum depth of any fraction of the inputs. For
a fraction of $1/n$, we have
\[
  m \ge \lg \frac{n!}{n} = \lg n! - \lg n = \Omega(n \lg n),
\]
and for a fraction of $1/2^n$, we have
\[
  m \ge \lg \frac{n!}{2^n} = \lg n! - \lg 2^n = \lg n! - n = \Omega(n \lg n).
\]
\end{framed}

\item[8.1{-}4]{Suppose that you are given a sequence of $n$ elements to sort.
The input sequence consists of $n/k$ subsequences, each containing $k$ elements.
The elements in a given subsequence are all smaller than the elements in the
succeeding subsequence and larger than the elements in the preceding
subsequence. Thus, all that is needed to sort the whole sequence of length
$n$ is to sort the $k$ elements in each of the $n/k$ subsequences. Show an
$\Omega(n \lg k)$ lower bound on the number of comparisons needed to solve this
variant of the sorting problem. (\emph{Hint}: It is not rigorous to simply
combine the lower bounds for the individual subsequences.)}

\begin{framed}
All we know is the ordering of the elements of a given subsequence with respect
to the elements of the previous/next subsequence. Thus, for each subsequence,
we have $k!$ possible permutations. Since there are $n/k$ input subsequences,
the number of possible outcomes for this sorting problem is
\[
  \prod_{i = 1}^{n/k} k! = k!^{(n/k)}.
\]

We can use here the same argument used in the text book to prove a lower bound
for any comparison sort algorithm. However, in this case, the number of possible
permutations is $k!^{(n/k)}$, instead of $n!$.  Thus, we need to show that the
height of any decision tree with at least $k!^{(n/k)}$ leaves is
$\Omega(n \lg k)$.  We have
\[
  k!^{n/k} \le l \le 2^h,
\]
which implies
\begin{equation*}
\begin{aligned}
h &\ge \lg \left( k!^{(n/k)} \right)\\
  &=   \frac{n}{k} \cdot \lg k!\\
  &=   \frac{n}{k} \cdot \sum_{i = 1}^{k} \lg i\\
  &=   \frac{n}{k} \cdot \sum_{i = 1}^{\floor{k/2}} \lg i + \frac{n}{k} \cdot \sum_{i = \floor{k/2} + 1}^{k} \lg i\\
  &\ge \frac{n}{k} \cdot \sum_{i = \floor{k/2}}^{k} \lg i\\
  &\ge \frac{n}{k} \cdot \left( \frac{k}{2} \lg \frac{k}{2} \right)\\
  &=   \frac{n}{2} \lg \frac{k}{2}\\
  &=   \Omega(n \lg k).\\
\end{aligned}
\end{equation*}
\end{framed}

\end{enumerate}

\end{document}
