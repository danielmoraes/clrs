\lhead{CLRS {--} Chapter 5 {--} Probabilistic Analysis and Randomized Algorithms}

{\large Section 5.1 {--} The hiring problem}

\begin{enumerate}

\item[5.1{-}1]{Show that the assumption that we are always able to determine
which candidate is best, in line 4 of procedure \textsc{Hire-Assistant}, implies
that we know a total order on the ranks of the candidates.}

\begin{framed}
Let $A$ be the set of candidates in random order and $R$ the binary relation
``is better than or equal'' on the set $A$. $R$ is a total order if
\begin{enumerate}
  \item $R$ is \textbf{reflexive}. That is, $a\;R\;a \; \Forall a \in A$;
  \item $R$ is \textbf{antisymmetric}. That is, $a\;R\;b$ and $b\;R\;a$ imply $a = b$;
  \item $R$ is \textbf{transitive}. That is, $a\;R\;b$ and $b\;R\;c$ imply $a\;R\;c$;
  \item $R$ is a \textbf{total relation}. That is, $a\;R\;b$ or $b\;R\;a \; \Forall a, b \in A$.
\end{enumerate}

The above properties are necessary because
  \begin{enumerate}
    \item if two different candidates have the same qualification, it is
      necessary so that they can be compared;
    \item if both $a$ is ``better than or equal'' than $b$ and $b$ is ``better
      than or equal'' than $a$ and they qualifications are not equal, we would
      not be able to choose one of them and still be hiring ``the best candidate
      we have seen so far'';
    \item if we hire $b$ because he is ``better than or equal'' than $a$ and
      then we hire $c$ because he is ``better than or equal'' than $b$ and $c$
      is not ``better than or equal'' than $a$, we are not hiring ``the best
      candidate we have seen so far'';
    \item if $R$ is not a total relation, we would not be able to compare any
      two candidates.
\end{enumerate}
\end{framed}

\item[5.1{-}2]{($\star$) Describe an implementation of the procedure \textsc{Random}(a, b)
that only makes calls to \textsc{Random}(0, 1). What is the expected running
time of your procedure, as a function of a and b?}

\begin{framed}
The pseudocode is stated below.\\
\begin{algorithm}[H]
\SetAlgoNoEnd\DontPrintSemicolon
\BlankLine
\SetKwFunction{algo}{RandomInterval}
\SetKwProg{myalg}{}{}{}
\myalg{\algo{a, b}}{%
\nl $flips = \ceil{\lg (b - a)}$\;
\nl $count = \infty$\;
\nl \While{$count > b$}{%
\nl   $count = 0$\;
\nl   \For{$i = 1$ \KwTo $flips$}{%
\nl     $count = count + (2^{i - 1} \cdot \texttt{Random}(0, 1))$\; } }
\nl \Return{$count + a$}\; }
\end{algorithm}

The expected running time is
\[
  \underbrace{2^{\ceil{\lg(b - a)}}/(b - a)}_\text{while loop} \cdot \underbrace{\ceil{\lg(b - a)}}_\text{for loop} < 2 \cdot \ceil{\lg(b - a)},
\]
where the last inequality is valid since $1 \le 2^{\ceil{\lg(b - a)}}/(b - a) < 2$.
\end{framed}

\item[5.1{-}3]{($\star$) Suppose that you want to output 0 with probability
$1/2$ and 1 with probability $1/2$.  At your disposal is a procedure
\textsc{Biased-Random}, that outputs either 0 or 1. It outputs 1 with some
probability $p$ and 0 with probability $1 - p$, where $0 < p < 1$, but you do
not know what $p$ is. Give an algorithm that uses \textsc{Biased-Random} as
a subroutine, and returns an unbiased answer, returning 0 with probability $1/2$
and 1 with probability $1/2$. What is the expected running time of your
algorithm as a function of $p$?}

\begin{framed}
The pseudocode is stated below.\\
\begin{algorithm}[H]
\SetAlgoNoEnd\DontPrintSemicolon
\BlankLine
\SetKwFunction{algo}{Random}
\SetKwProg{myalg}{}{}{}
\myalg{\algo{}}{%
\nl \While{$1$}{%
\nl   $r_1 = \texttt{Random}(0, 1)$\;
\nl   $r_2 = \texttt{Random}(0, 1)$\;
\nl   \If{$r_1 \neq r_2$}{%
\nl     \Return{$r_1$}\; } } }
\end{algorithm}

The expected running time is
\[
  \frac{1}{\underbrace{(1 - p) p}_\text{$(r_1, r_2) = (0, 1)$} + \underbrace{p (1 - p)}_\text{$(r_1, r_2) = (1, 0)$}} \cdot 1 = \frac{1}{2p (1- p)}.
\]
\end{framed}

\end{enumerate}

\newpage

{\large Section 5.2 {--} Indicator random variables}

\begin{enumerate}

\item[5.2{-}1]{In \textsc{Hire-Assistant}, assuming that the candidates are
presented in a random order, what is the probability that you hire exactly
one time? What is the probability that you hire exactly $n$ times?}

\begin{framed}
Since the initial dummy candidate is the least qualified,
\textsc{Hire-Assistant} will always hire the first candidate. It hires exacly
one time when the best candidate is the first to be interviewed. Thus, the
probability is $1/n$. To hire exactly $n$ times, the candidates has to be in
increasing order of quality. Since there are $n!$ possible orderings (each one
with equal probability of happening), the probability is $1/n!$.
\end{framed}

\item[5.2{-}2]{In \textsc{Hire-Assistant}, assuming that the candidates are
presented in a random order, what is the probability that you hire exactly
twice?}

\begin{framed}
The first candidate is always hired, thus the best qualified candidate cannot be
the first to be interviewed. Also, among all the candidates that are better
qualified than the first candidate, the best candidate must be interviewed first.
Otherwise, a third candidate will be hired between them. Now assume that the
first candidate to be interviewed is the $i$th best qualified, for $i = 2,
\dots, n$. This occurs with a probability of $1/n$. To hire exactly twice, the
best candidate must be the first to be interviewed among the $i - 1$ candidates
that are better qualified than candidate $i$. This occurs with a probability
of $1/(i - 1)$. Thus, the probability of hiring exactly twice is
\[
  \sum_{i = 2}^{n} \frac{1}{n} \frac{1}{i - 1} = \frac{1}{n} \sum_{i = 1}^{n - 1} \frac{1}{i} = \frac{1}{n} (\lg(n - 1) + O(1)).
\]
\end{framed}

\item[5.2{-}3]{Use indicator random variables to compute the expected value of
the sum of $n$ dice.}

\begin{framed}
Let $X_i$ be an indicator random variable of a dice coming up the number $i$. We
have $\text{Pr}\{X_i\} = 1/6$. Let $X$ be a random variable denoting the
result of throwing a dice. Then
\[
  \text{E}[X] = \sum_{i = 1}^{6} i \cdot \text{Pr}\{X_i\}
              = \sum_{i = 1}^{6} i \cdot \frac{1}{6}
              = \frac{1}{6} \sum_{i = 1}^{6} i
              = \frac{1}{6} \frac{6 \cdot 7}{2} = 3.5.
\]
By linearity of expectations, the expected value of the sum of $n$ dice is the
sum of the expected value of each dice. Thus,
\[
  \sum_{i = 1}^{n} \text{E}[X] = \sum_{i = 1}^{n} 3.5 = 3.5 \cdot n.
\]
\end{framed}

\item[5.2{-}4]{Use indicator random variables to solve the following problem,
which is known as the \textbf{\emph{hat-check problem}}. Each of $n$ customers
gives a hat to a hat-check person at a restaurant. The hat-check person
gives the hats back to the customers in a random order. What is the expected
number of customers who get back their own hat?}

\begin{framed}
Let $X_i$ be an indicator random variable of customer $i$ getting back his own
hat. We have
\[
\text{Pr}\{X_i\} = \text{E}[X_i] = 1/n.
\]
Let $X$ be a random variable denoting the number of customers who get back their
own hat. Then
\[
  X = X_1 + X_2 + \cdots + X_n,
\]
which implies
\begin{equation*}
\begin{aligned}
  \text{E}[X] &= \text{E}\left[\sum_{i = 1}^{n} X_i\right]\\
              &= \sum_{i = 1}^{n} \text{E}[X_i]\\
              &= \sum_{i = 1}^{n} \frac{1}{n}\\
              &= 1.
\end{aligned}
\end{equation*}
\end{framed}

\item[5.2{-}5]{Let $A[1, \dots, n]$ be an array of $n$ distinct numbers. If
$i < j$ and $A[i] > A[j]$, then the pair $(i, j)$ is called an
$\textbf{\emph{inversion}}$ of $A$. (See Problem 2-4 for more on inversions.)
Suppose that the elements of $A$ form a uniform random permutation of
$\langle 1, 2, \dots, n \rangle$. Use indicator random variables to compute
the expected number of inversions.}

\begin{framed}
Let $X_{ij}$ be an indicator random variable for the event that the pair
$(i, j)$ is inverted. Since $A$ forms a uniform random permutation, we have
\[
\text{Pr}\{X_{ij}\} = \text{Pr}\{\overline{X_{ij}}\} = 1/2,
\]
which implies
\[
  \text{E}[X_{ij}] = 1/2.
\]
Let $X$ be a random variable denoting the number of
inversions of $A$. Since there are $\binom{n}{2}$ possible pairs on $A$, each
with probability $1/2$ of being inverted, we have
\[
  \text{E}[X] = \binom{n}{2} \frac{1}{2} = \frac{n!}{2! \cdot (n - 2)!} \frac{1}{2} = \frac{n (n - 1)}{4}.
\]
\end{framed}

\end{enumerate}

\newpage

{\large Section 5.3 {--} Randomized algorithms}

\begin{enumerate}

\item[5.3{-}1]{Professor Marceau objects to the loop invariant used in the proof
of Lemma 5.5. He questions whether it is true prior to the first iteration. He
reasons that we could just as easily declare that an empty subarray contains
no 0-permutations. Therefore, the probability that an empty subarray
contains a 0-permutation should be 0, thus invalidating the loop invariant
prior to the first iteration. Rewrite the procedure \textsc{Randomize-In-Place}
so that its associated loop invariant applies to a nonempty subarray prior
to the first iteration, and modify the proof of Lemma 5.5 for your procedure.}

\begin{framed}
Just select a random element in the array and swap it with the first element.

\begin{algorithm}[H]
\SetAlgoNoEnd\DontPrintSemicolon
\BlankLine
  \SetKwFunction{algo}{Randomize-In-Place}
\SetKwProg{myalg}{}{}{}
\myalg{\algo{A}}{%
\nl $n = A.length$\;
\nl swap $A[1]$ with $A[\texttt{Random}(1, n)]$\;
\nl \For{$i = 2$ \KwTo $n - 1$}{%
\nl   swap $A[i]$ with $A[\texttt{Random}(i, n)]$\; } }
\end{algorithm}

The only difference in the proof of Lemma 5.5 is the initialization of the loop
invariant:
\begin{itemize}
  \item \textbf{Initialization.} Consider the situation just before the first
    loop iteration, so that $i = 2$. The loop invariant says that for each
    possible 1-permutation, the subarray $A[1, \dots, 1]$ contains this
    1-permutation with probability $(n - i + 1)/n! = (n - 1)!/n! = 1/n$. The
    subarray $A[1, \dots, 1]$ has a single element and this element was
    randomly choosed among the $n$ elements of the array. Thus, $A[1, \dots, 1]$
    contains this $1$-permutation with probability $1/n$, and the loop invariant
    holds prior to the first iteration.
\end{itemize}
\end{framed}

\item[5.3{-}2]{Professor Kelp decides to write a procedure that produces at
random any permutation besides the identity permutation. He proposes the
following procedure:

\begin{algorithm}[H]
\SetAlgoNoEnd\DontPrintSemicolon
\BlankLine
\SetKwFunction{algo}{Permute-Without-Identity}
\SetKwProg{myalg}{}{}{}
\myalg{\algo{A}}{%
\nl $n = A.length$\;
\nl \For{$i = 1$ \KwTo $n - 1$}{%
\nl   swap $A[i]$ with $A[\texttt{Random}(i + 1, n)]$\; } }
\end{algorithm}

Does this code do what Professor Kelp intends?
}

\begin{framed}
No. This code enforces that every position $i$ of the resulting array receives
an element that is different from the $i$th element of the original array.
However, this requirement discards much more permutations than just the identity
permutation. For instance, consider the array $A = [1, 2, 3]$ and a permutation
of it $A' = [1, 3, 2]$. In this case, the permutation $A'$ is not identical
to the original array $A$. However, Professor Kelp's code is not able to produce
this permutation.
\end{framed}

\item[5.3{-}3]{Suppose that instead of swapping element $A[i]$ with a random
element from the subarray $A[i, \dots, n]$, we swapped it with a random element
from anywhere in the array:

\begin{algorithm}[H]
\SetAlgoNoEnd\DontPrintSemicolon
\BlankLine
\SetKwFunction{algo}{Permute-With-All}
\SetKwProg{myalg}{}{}{}
\myalg{\algo{A}}{%
\nl $n = A.length$\;
\nl \For{$i = 1$ \KwTo $n$}{%
\nl   swap $A[i]$ with $A[\texttt{Random}(1, n)]$\; } }
\end{algorithm}

Does this code produce a uniform random permutation? Why or why not?
}

\begin{framed}
No. As a counterexample, consider the input array $A = [1, 2, 3]$. Since each
call to \textsc{Random} can produce one of three values, the number of possible
outcomes after all the \textsc{Random} calls can be seen as the number of
strings over the set $\{1, 2, 3\}$, which is $3^3 = 27$. However, since an array
of size $3$ has $3! = 6$ distinct permutations, and 27 is not divisible by 6, it
is not possible that each of the 6 permutations of $A$ has the same probability
of happening among the 27 possible outcomes of \textsc{Permute-With-All}.
\end{framed}

\newpage

\item[5.3{-}4]{Professor Armstrong suggests the following procedure for
generating a uniform random permutation:

\begin{algorithm}[H]
\SetAlgoNoEnd\DontPrintSemicolon
\BlankLine
\SetKwFunction{algo}{Permute-By-Cyclic}
\SetKwProg{myalg}{}{}{}
\myalg{\algo{A}}{%
\nl $n = A.length$\;
\nl let $B[1 \dots n]$ be a new array\;
\nl \emph{offset} $= \texttt{Random}(1, n)$\;
\nl \For{$i = 1$ \KwTo $n$}{%
\nl   $\text{\emph{dest}} = i + \text{\emph{offset}}$\;
\nl   \If{\emph{dest} $> n$}{%
\nl     \emph{dest} $=$ \emph{dest} $- n$\; }
\nl   $B[\text{\emph{dest}}] = A[i]$\; } }
\end{algorithm}

Show that each element $A[i]$ has a $1/n$ probability of winding up in any
particular position in $B$. Then show that Professor Armstrong is mistaken by
showing that the resulting permutation is not uniformly random.
}

\begin{framed}
What Professor Armstrong's code does is a circular shift of all the elements to
the right by $i$ positions. Since each of the $n$ possible shifts has the same
probability of happening, each element has indeed a probability of $1/n$ of
winding up in any particular position of the final array $B$. However, since
this code has only $n$ possible outcomes and $A$ has $n!$ permutations, it can
not produce a uniform random distribution over $A$. More precisely, the
Professor Armstrong's code is not able to produce any permutation of $A$ that
is not a circular shift of $A$.
\end{framed}

\item[5.3{-}5]{($\star$) Prove that in the array $P$ in procedure
\textsc{Permute-By-Sorting}, the probability that all elements are unique is at
least $1 - 1/n$.}

\begin{framed}
Let $X_i$ be an indicador random variable for the event that the $i$th priority
is not unique. Since the subarray $P[1, \dots, i - 1]$ has at most $i - 1$
distinct numbers, we have $\text{Pr}\{X_i\} = \text{E}[X_i] \le (i - 1)/n^3$.
Let $X$ be a random variable for the event that at least one priority is not
unique. Then
\[
  X = (X_1 \cup X_2 \cup \cdots X_n) = X_1 + X_2 + \cdots + X_n,
\]
which implies
\begin{equation*}
\begin{aligned}
  \text{E}[X] &=   \text{E}\left[\sum_{i = 1}^{n} X_i\right]\\
              &=   \sum_{i = 1}^{n} \text{E}[X_i]\\
              &\le \sum_{i = 1}^{n} \frac{i - 1}{n^3}\\
              &=   \frac{1}{n^3} \sum_{i = 0}^{n - 1} i\\
              &=   \frac{1}{n^3} \frac{(n - 1) \cdot n}{2}\\
              &=   \frac{n - 1}{2 n^2}\\
              &\le \frac{1}{n}.
\end{aligned}
\end{equation*}

Thus, the probability that all elements are unique is
\[
  \text{E}[\overline{X}] = 1 - \text{E}[X] \ge 1 - \frac{1}{n}.
\]
\end{framed}

\newpage

\item[5.3{-}6]{Explain how to implement the algorithm
\textsc{Permute-By-Sorting} to handle the case in which two or more priorities
are identical. That is, your algorithm should produce a uniform random
permutation, even if two or more priorities are identical.}

\begin{framed}
The pseudocode is stated below.

\begin{algorithm}[H]
\SetAlgoNoEnd\DontPrintSemicolon
\BlankLine
\SetKwFunction{algo}{Permute-By-Sorting-Unique}
\SetKwProg{myalg}{}{}{}
\myalg{\algo{A}}{%
\nl $n = A.length$\;
\nl let $P[1 \dots n]$ be a new array\;
\nl \Repeat{\text{unique}}{%
\nl   \For{$i = 1$ \KwTo $n$}{%
\nl     $P[i] = \texttt{Random}(1, n^3)$\; }
\nl   let $Q$ be a copy of $P$\;
\nl   sort $Q$\;
\nl   $\text{\emph{unique}} = \text{True}$\;
\nl   \For{$i = 2$ \KwTo $n$}{%
\nl     \If{$Q[i] == Q[i - 1]$} {%
\nl       $\text{\emph{unique}} = \text{False}$\;
\nl       break\; } } }
\nl sort $A$, using $P$ as sort keys\; }
\end{algorithm}

Before sorting $A$ using $P$ as sort keys, the above algorithm verifies if $P$
has unique priorities. If the priorities are not unique, $P$ is generated again
until it has unique priorities. Since the probability that a random $P$ is
unique is at least $1 - 1/n$, the expected number of iterations of the repeat
loop of lines 3-12 is less than $2$.
\end{framed}

\newpage

\item[5.3{-}7]{Suppose we want to create a \textbf{\emph{random sample}} of the
set $\{1, 2, 3, \dots, n\}$, that is, an $m$-element subset $S$, where
$0 \le m \le n$, such that each $m$-subset is equally likely to be created. One
way would be to set $A[i] = i$ for $i = 1, 2, 3, \dots, n$, call
\textsc{Randomized-In-Place}($A$), and then take just the first $m$ array
elements. This method would make $n$ calls to the \textsc{Random} procedure. If
$n$ is mucn larger than $m$, we can create a random sample with fewer calls to
\textsc{Random}. Show that the following recursive procedure returns a random
$m$-subset $S$ of $\{1, 2, 3, \dots, n\}$, in which each $m$-subset is equally
likely, while making only $m$ calls to \textsc{Random}:

\begin{algorithm}[H]
\SetAlgoNoEnd\DontPrintSemicolon
\BlankLine
\SetKwFunction{algo}{Random-Sample}
\SetKwProg{myalg}{}{}{}
\myalg{\algo{m, n}}{%
\nl \If{$m == 0$}{%
\nl   \Return{$\emptyset$}\; }
\nl  \Else{%
\nl    $S = \texttt{Random-Sample}(m - 1, n - 1)$\;
\nl    $i = \texttt{Random}(1, n)$\;
\nl    \If{$i \in S$}{%
\nl      $S = S \cup \{n\}$\; }
\nl    \Else{%
\nl      $S = S \cup \{i\}$\; }
\nl    \Return{S}\; } }
\end{algorithm}
}

\begin{framed}
The recursion has $m + 1$ levels. Let $R_k$, for $k = 0, 1, \dots, m$, denote
the recursion at depth $k$, in which an $k$-subset is returned ($R_0$ returns
the empty set; $R_m$ returns the final $m$-subset). After $R_k$, $S$ will
consist of $k$ elements from the set $\{1, 2, \dots, n - (m - k)\}$. There are
$\binom{n - (m - k)}{k}$ ways to choose $k$ elements from an
$(n - (m - k))$-set. Thus, to $S$ be a random sample, we wish to show that, in
each recursion level $k$, this particular $k$-subset is selected with
probability $1/\binom{n - (m - k)}{k}$.

For the base case of the recursion, which occurs when $k = 0$, there are
$\binom{n - m}{0} = 1$ distincts $0$-subsets and the algorithm returns the empty
set with probability $1 = 1/\binom{n - m}{0}$. Now assume $R_{k - 1}$ returns an
random $(k - 1)$-sample. There are two ways to add the $k$th element to $S$ on
$R_k$:
\begin{itemize}
\item The element $n - (m - k)$ is added. This occurs when line 5 either
selects the element $n - (m - k)$ or an element $e$ such that $e \in R_{k - 1}$.
This probability is
\[
  \underbrace{\frac{1}{n - (m - k)}}_\text{$(n - (m - k))$ is selected} +
  \underbrace{\frac{k - 1}{n - (m - k)}}_\text{$e \in R_{k - 1}$ is selected} = \frac{k}{n - (m - k)}.
\]
Thus, $R_k$ produces a particular $k$-sample with the element $n - (m - k)$ with
probability
\begin{equation*}
\begin{aligned}
  \frac{k}{n - (m - k)} \cdot \frac{1}{\binom{n - (m - k) - 1}{k - 1}}
  &= \frac{k}{n - (m - k)} \cdot \left(\frac{(n - (m - k) - 1)!}{(k - 1)! \cdot (n - (m - k) - 1 - (k - 1))}\right)^{-1}\\
  &= \left(\frac{(n - (m - k))!}{k! \cdot (n - (m - k) - k )}\right)^{-1}\\
  &= \frac{1}{\binom{n - (m - k)}{k}}.
\end{aligned}
\end{equation*}

\item An element $j < n - (m - k)$ is added. The probability of line 5 selecting
such element is
\[
  \frac{n - (m - k) - k}{n - (m - k)} = \frac{n - m}{n - (m - k)}.
\]
Thus, $R_k$ produces a particular $k$-sample with the element $j$ with
probability
\begin{equation*}
\begin{aligned}
  \frac{n - m}{n - (m - k)} \cdot \frac{1}{\binom{n - (m - k) - 1}{k}}
  &= \frac{n - m}{n - (m - k)} \cdot \left(\frac{(n - (m - k) - 1)!}{k! \cdot (n - (m - k) - 1 - k)}\right)^{-1}\\
  &= \left(\frac{(n - (m - k))!}{k! \cdot (n - (m - k) - k)}\right)^{-1}\\
  &= \frac{1}{\binom{n - (m - k)}{k}}.
\end{aligned}
\end{equation*}
\end{itemize}

Since each recursion level $R_k$ such that $k > 0$ makes exactly one
call to \textsc{Random}, there are $m$ such calls. Also, among the
$\binom{n}{m}$ ways of choosing $m$ elements from an $n$-set,
\textsc{Random-Sample} returns each of them with probability
\[
  \frac{1}{\binom{n - (m - m)}{m}} = \frac{1}{\binom{n}{m}}.
\]
\end{framed}

\end{enumerate}

\newpage

{\large Problems}

\begin{enumerate}
\item[5{-}1]{\textbf{\emph{Probabilistic counting}}\\
With a $b$-bit counter, we can ordinarily only count up to $2^b - 1$. With R.
Morri's \textbf{\emph{probabilistic counting}}, we can count up to a much larger
value at the expense of some loss of precision.

We let a counter value of $i$ represent a count of $n_i$ for
$i = 0, 1, \dots, 2^b - 1$, where the $n_i$ form an increasing sequence of
nonnegative values. We assume that the initial value of the counter is 0,
representing a count of $n_0 = 0$. The \textsc{Increment} operation works on
a counter containing the value $i$ in a probabilistic manner. If $i = 2^b - 1$,
then the operation reports an overflow error. Otherwise, the \textsc{Increment}
operation increases the counter by 1 with probability $1/(n_{i + 1} - n_i)$.

If we select $n_i = i$ for all $i \ge 0$, then the counter is an ordinary one.
More interesting situations arise if we select, say, $n_i = 2^{i - 1}$ for
$i > 0$ or $n_i = F_i$ (the $i$th Fibonacci number {--} see Section 3.2).

For this problem, assume $n_{2^b - 1}$ is large enough that the probability of
an overflow error is negligible.

\begin{enumerate}
\item[a.] Show that the expected value represented by the counter after $n$
\textsc{Increment} operations have been performed is exactly $n$.
\item[b.] The analysis of the variance of the count represented by the counter
depends on the sequence of the $n_i$. Let us consider a simple case:
$n_i = 100i$ for all $i \ge 0$. Estimate the variance in the value represented
by the register after $n$ \textsc{Increment} operations have been performed.
\end{enumerate}
}

\begin{framed}
\begin{enumerate}
\item
Let $X_i$ denote a random variable for the expected \emph{increment} of the
count represented by a counter of value $i$ after \emph{one} \textsc{Increment}
operation. We have
\[
  \text{E}[X_i] = 0 \cdot \left(1 - \frac{1}{n_{i + 1} - n_{i}}\right) + (n_{i + 1} - n_{i}) \cdot \frac{1}{n_{i + 1} - n_{i}} = 1,
\]
which shows that, independently from the current state of the \emph{counter},
the expected \emph{increment} of the \emph{count} after each \textsc{Increment}
operation is always 1. Thus, after $n$ \textsc{Increment} operations, the
expected \emph{count} is:
\[
  \sum_{i = 1}^{n} \text{E}[X_0] = \sum_{i = 1}^{n} 1 = n,
\]

\item We have
\begin{equation*}
\begin{aligned}
  \text{Var}[X_i] &= \text{E}[{X_i}^2] - \text{E}^2[X_i]\\
                  &= \left( 0^2 \cdot \left(1 - \frac{1}{100}\right) + 100^2 \cdot \frac{1}{100} \right) - 1\\
                  &= 99,
\end{aligned}
\end{equation*}
which shows that the estimated variance after each \textsc{Increment} operation
does not depend on the current state of the \emph{counter}. Thus, after $n$
\textsc{Increment} operations, the
estimated variance is
\[
  \sum_{i = 1}^{n} \text{Var}[X_0] = \sum_{i = 1}^{n} 99 = 99n.
\]
\end{enumerate}
\end{framed}

\newpage

\item[5{-}2]{\textbf{\emph{Searching an unsorted array}}\\
This problem examines three algorithms for searching for a value $x$ in an
unsorted array $A$ consisting of $n$ elements.

Consider the following randomized strategy: pick a random index $i$ into $A$. If
$A[i] = x$, then we terminate; otherwise, we continue the search by picking
a new random index into $A$. We continue picking random indices into $A$ until
we find an index $j$ such that $A[j] = x$ or until we have checked every element
of $A$. Note that we pick from the whole set of indices each time, so that we
may examine a given element more than once.

\begin{enumerate}
\item[\textbf{a.}] Write pseudocode for a procedure \textsc{Random-Search} to
implement the strategy above. Be sure that your algorithm terminates when all
indices into $A$ have been picked.
\item[\textbf{b.}] Suppose that there is exactly one index $i$ such that
$A[i] = x$. What is the expected number of indices into $A$ that we must pick
before we find $x$ and \textsc{Random-Search} terminates?
\item[\textbf{c.}] Generalizing your solution to part (b), suppose that there
are $k \ge 1$ indices $i$ such that $A[i] = x$. What is the expected number of
indices into $A$ that we must pick before we find $x$ and \textsc{Random-Search}
terminates? Your answer should be a function of $n$ and $k$.
\item[\textbf{d.}] Suppose that there are no indices $i$ such that $A[i] = x$.
What is the expected number of indices into $A$ that we must pick before we
have checked all elements of $A$ and \textsc{Random-Search} terminates?
\end{enumerate}

Now consider a deterministic linear serach algorithm, which we refer to as
\textsc{Deterministic-Search}. Specifically, the algorithm searches $A$ for $x$
in order, considering $A[1], A[2], A[3], \dots, A[n]$ until either it finds
$A[i] = x$ or it reaches the end of the array. Assume that all possible
permutations of the input array are equally likely.

\begin{enumerate}
\item[\textbf{e.}] Suppose that there is exactly one index $i$ such that
$A[i] = x$. What is the average-case running time of
\textsc{Deterministic-Search}?  What is the worst-case running time of
\textsc{Deterministic-Search}?
\item[\textbf{f.}] Generalizing your solution to part (e), suppose that there
are $k \ge 1$ indices $i$ such that $A[i] = x$. What is the average-case running
time of \textsc{Deterministic-Search}? What is the worst-case running time of
\textsc{Deterministic-Search}? Your answer should be a function of $n$ and $k$.
\item[\textbf{g.}] Suppose that there are no indices $i$ such that $A[i] = x$.
What is the average-case running time of \textsc{Deterministic-Search}? What
is the worst-case running time of \textsc{Deterministic-Search}?
\end{enumerate}

Finally, consider a randomized algorithm \textsc{Scramble-Search} that works by
first randomly permuting the input array and then running the deterministic
linear search given above on the resulting permuted array.

\begin{enumerate}
\item[\textbf{h.}] Letting $k$ be the number of indices $i$ such that
$A[i] = x$, give the worst-case and expected running times of
\textsc{Scramble-Search} for the cases in which $k = 0$ and $k = 1$. Generalize
your solution to handle the case in which $k \ge 1$.
\item[\textbf{i.}] Which of the three searching algorithms would you use?
Explain your answer.
\end{enumerate}
}

\begin{framed}
\begin{enumerate}
\item The pseudocode is stated below.

\begin{algorithm}[H]
\SetAlgoNoEnd\DontPrintSemicolon
\BlankLine
\SetKwFunction{algo}{Random-Search}
\SetKwProg{myalg}{}{}{}
\myalg{\algo{A, x}}{%
\nl $I = \emptyset$\;
\nl $n = A.length$\;
\nl $\text{\emph{index}} = -1$\;
\nl \While{$|I| < n$}{%
\nl   $i = \texttt{Random}(1, n)$\;
\nl   $I = I \cup \{i\}$\;
\nl   \If{$A[i] == x$}{%
\nl     $\text{\emph{index}} = i$\;
\nl     break\; } }
\nl \Return{index}\; }
\end{algorithm}

\item This can be viewed as a sequence of Bernoulli trials, each with
a probability $p = 1/n$ of success. Let $X$ be a random variable for the number
of trials needed to pick $i$ such that $A[i] = x$. From Equation (C.32), we have
\[
  \text{E}[X] = \frac{1}{p} = n.
\]

\item This can also be viewed as a sequence of Bernoulli trials, but with
a probability $p = k/n$ of success. Thus, we have
\[
  \text{E}[X] = \frac{1}{p} = \frac{n}{k}.
\]

\item Let $I$ be the set of indexes that was already checked. Let $X_i$ be
a random variable for the number of trials needed to pick an index $i$, for
$i = 1, 2, \dots, n$, such that $i \notin I$ and $|I| = i - 1$. This can be
viewed as a sequence of Bernoulli trials. Thus, we have
\[
  p = \frac{n - |I|}{n} = \frac{n - i + 1}{n},
\]
and
\[
  \text{E}[X_i] = \frac{1}{p} = \frac{n}{n - i + 1}.
\]

Now let $X$ be a random variable for the number of trials to pick all elements
of $A$. We have
\begin{equation*}
\begin{aligned}
  \text{E}[X] &= \text{E}\left[\sum_{i = 1}^{n} X_i \right]
              = \sum_{i = 1}^{n} \text{E}[X_i]\\
              &= \sum_{i = 1}^{n} \frac{n}{n - i + 1}\\
              &= n \sum_{i = 1}^{n} \frac{1}{n - i + 1}\\
              &= n \sum_{i = 0}^{n - 1} \frac{1}{n - i}\\
              &= n \sum_{i = 1}^{n} \frac{1}{i} & \text{($n$th harmonic number)}\\
              &= n (\ln n + O(1)).
\end{aligned}
\end{equation*}

\item Lets first consider the average case. Among the $n - 1$ elements that is
not $x$, $(n - 1)/2$ of them are expected to be before the element $x$ on the
array. Thus, the expected running time of the algorithm is
\[
  \frac{n - 1}{2} + 1 = \frac{n + 1}{2}.
\]

The worst-case occur when the number of elements before $x$ is $n - 1$. In this
case, the algorithm will make $n$ checks.

\item Let $I$ be the set of indexes such that $i \in I \rightarrow A[i] = x$.
For each element $e$ such that $e \neq x$, there are $k + 1$ possibilities to
position $e$ with respect to $I$ (before all elements of $I$, after one
element of $I$, but before the remaining $k - 1$ elements of $I$, and so on).
Each of these positions is equally likely.  Therefore, among the $n - k$
elements that is not $x$, $(n - k) \cdot 1/(k + 1) = (n - k)/(k + 1)$ are
expected to be before all the elements of $I$. Thus, the expected running time
of the algorithm is
\[
  \frac{n - k}{k + 1} + 1 = \frac{(n - k) + (k + 1)}{k + 1} = \frac{n + 1}{k + 1}.
\]

The worst-case occurs when the number of elements before the first $x$ is
$n - k$. In this case, the algorithm will make $n - k + 1$ checks.

\item In every case, the algorithm will check all elements of $A$. Thus, there
will be $n$ checks.

\item Suppose the algorithm uses \textsc{Randomize-In-Place} to randomize the
input array. Independently from the value of $k$, the algorithm will take
$n$ on this operation. Thus, lets focus on the number of checks for each
case. When $k = 0$, the algorithm will make exactly $n$ checks in every case.
Thus, it the expected running time is $n + n = 2n$. When $k = 1$, the
behaviour of the algorithm is similar to the one of item (e). Thus, the expected
running time is $n + (n + 1)/2 = (3n + 1)/2$. As for the worst-case, note that
this notation refers to the distribution of inputs.  Since for every input the
expected running time is the same, the worst-case (over the inputs) is
$n + (n + 1)/2 = (3n + 1)/2$. Similarly, for a given $k$ and from item (f),
both the expected running time and the worst-case is $n + (n + 1)/(k + 1)$.

\item \textsc{Deterministic-Search} is better in all cases.
\end{enumerate}
\end{framed}

\end{enumerate}
