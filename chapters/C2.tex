\documentclass{report}

%                       MAIN PACKAGES                       %
% --------------------------------------------------------- %

\usepackage[english,brazil,brazilian]{babel}
\usepackage{graphicx}
\usepackage{caption}
\usepackage{isolatin1}
\usepackage{amsmath}
\usepackage{amssymb}
\usepackage{textcomp}
\usepackage[usenames,dvipsnames]{color}
\usepackage{soul}
\usepackage{cancel}
\usepackage{booktabs}
\usepackage{multirow}
\usepackage{framed}
\usepackage{xcolor}
\usepackage{algpseudocode}
\usepackage{algorithm2e} %for psuedo code
\usepackage{courier}
\usepackage{mathtools}

\usepackage{tikz}
\usetikzlibrary{calc}

%                         MARGINS                           %
% --------------------------------------------------------- %

\hoffset=-0.5in
\voffset=-0.6in
\oddsidemargin=0pt
\topmargin=0pt
\headheight=12pt
\headsep=15pt
\textheight=690pt
\textwidth=543pt
\marginparsep=11pt
\marginparwidth=54pt
\footskip=25pt
\marginparpush=5pt
\paperwidth=597pt
\paperheight=845pt

%                        PAGE STYLE                         %
% --------------------------------------------------------- %

\usepackage{fancyhdr}
\pagestyle{fancy}
\lhead{CLRS {--} Chapter 2 {--} Getting Started}
\rhead{Daniel Bastos Moraes}
\cfoot{}
\rfoot{\thepage}
\renewcommand{\headrulewidth}{0.4pt}
\renewcommand{\footrulewidth}{0.4pt}
\newcommand{\Perp}{\perp\! \! \! \perp}

%                       PROPERTIES                          %
% --------------------------------------------------------- %

\setlength{\parindent}{0cm}

\makeatletter
\renewenvironment{framed}{%
 \def\FrameCommand##1{\hskip\@totalleftmargin
 \fboxsep=\FrameSep\fbox{##1}}%
 \MakeFramed {\advance\hsize-\width
   \@totalleftmargin\z@ \linewidth\hsize
   \@setminipage}}%
 {\par\unskip\endMakeFramed}
\makeatother

\makeatletter
\def\BState{\State\hskip-\ALG@thistlm}
\makeatother

\DeclarePairedDelimiter{\ceil}{\lceil}{\rceil}
\DeclarePairedDelimiter{\floor}{\lfloor}{\rfloor}

%                        DOCUMENT                           %
% --------------------------------------------------------- %

\begin{document}

\small

{\large Section 2.1 {--} Insertion sort}

\begin{enumerate}

\item[2.1{-}1]{Using Figure 2.2 as a model, illustrate the operation of
  \textsc{Insertion-Sort} on the array $A = \langle 31, 41, 59, 26, 41, 58 \rangle$.}

\begin{framed}
{\centering $ \displaystyle
\begin{aligned}
  \colorbox{white}{(\texttt{a})} \colorbox{white}{31} \colorbox{yellow}{\textcolor{red}{41}}
  \colorbox{white}{59} \colorbox{white}{26} \colorbox{white}{41} \colorbox{white}{58}\\
%
  \colorbox{white}{(\texttt{b})} \colorbox{white}{31} \colorbox{white}{41} \colorbox{yellow}{\textcolor{red}{59}}
  \colorbox{white}{26} \colorbox{white}{41} \colorbox{white}{58}\\
%
  \colorbox{white}{(\texttt{c})} \colorbox{gray!20}{\textcolor{red}{31}} \colorbox{gray!20}{41} \colorbox{gray!20}{59}
  \colorbox{yellow}{\textcolor{red}{26}} \colorbox{white}{41} \colorbox{white}{58}\\
%
  \colorbox{white}{(\texttt{d})} \colorbox{white}{26} \colorbox{white}{31} \colorbox{white}{41}
  \colorbox{gray!20}{\textcolor{red}{59}} \colorbox{yellow}{\textcolor{red}{41}} \colorbox{white}{58}\\
%
  \colorbox{white}{(\texttt{e})} \colorbox{white}{26} \colorbox{white}{31} \colorbox{white}{41}
  \colorbox{white}{41} \colorbox{gray!20}{\textcolor{red}{59}} \colorbox{yellow}{\textcolor{red}{58}}\\
%
  \colorbox{white}{(\texttt{f})} \colorbox{white}{26} \colorbox{white}{31} \colorbox{white}{41}
  \colorbox{white}{41} \colorbox{white}{58} \colorbox{white}{59}\\
\end{aligned} $ \par} % Necessary for centering to work
\end{framed}

\item[2.1{-}2]{Rewrite the \textsc{Insertion-Sort} procedure to sort into
  non-increasing instead of non-decreasing order.}


\begin{framed}
The pseudocode is stated below.\\
\begin{algorithm}[H]
\SetAlgoNoEnd\DontPrintSemicolon
\BlankLine
\SetKwFunction{algo}{InsertionSortNonIncreasing}
\SetKwProg{myalg}{}{}{}
\myalg{\algo{A}}{%
\nl \For{$j = 2$ \KwTo $A.length$}{%
\nl   $key = A[j]$\;
\nl   $i = j - 1$\;
\nl   \While{$i > 0$ \upshape{and} $A[i] > key$}{%
\nl     $A[i + 1] = A[i]$\;
\nl     $i = i - 1$\; }
\nl   $A[i + 1] = key$\; } }
\end{algorithm}
\end{framed}

\item[2.1{-}3]{Consider the \textbf{\emph{searching problem}}:\vspace{1mm}\\
\textbf{Input:} A sequence of $n$ numbers $A = \langle a_1, a_2, \dots, a_n \rangle$ and a value $\nu$.\\
\textbf{Output:} An index $i$ such that $\nu = A[i]$ or the special value NIL if $\nu$ does not appear in $A$.\vspace{1mm}\\
Write pseudocode for linear search, which scans through the sequence, looking
for $\nu$. Using a loop invariant, prove that your algorithm is correct. Make
sure that your loop invariant fulfills the three necessary properties.}

\begin{framed}
The pseudocode is stated below.\\
\begin{algorithm}[H]
\SetAlgoNoEnd\DontPrintSemicolon
\BlankLine
\SetKwFunction{algo}{LinearSearch}
\SetKwProg{myalg}{}{}{}
\myalg{\algo{A, $\nu$}}{%
  \nl \For{$i = 1$ \KwTo $A.length$}{%
  \nl   \If{$A[i] == \nu$}{%
  \nl     \Return{$i$}\; } }
  \nl \Return{\upshape{NIL}}\; }
\end{algorithm}

Here is the \emph{loop invariant}. At the start of each iteration of the
\textbf{for} loop of lines 1{--}3, the algorithm assures that the subarray $A[1,
\dots, i - i]$ does not contain the element $\nu$. Within each iteration, if
$A[i]$ corresponds to the $\nu$ element, its index is returned.

\begin{itemize}
  \item \textbf{Initialization.} Before the \textbf{for} loop, $i = 1$ and $A[1,
    \dots, i - 1]$ constains no element (therefore does not contain $\nu$).

  \item \textbf{Maintenance.} The body of the \textbf{for} loop verifies if
    $A[i]$ corresponds to the $\nu$ element. If the element correspond to $\nu$,
    its index is returned. Otherwise, incrementing $i$ for the next iteration of
    the \textbf{for} loop then preserves the loop invariant.

  \item \textbf{Termination.} The \textbf{for} loop can terminate in one of the
    following conditions: (1) $A[i] = \nu$, which means that $\nu$ was found and
    its index is returned; (2) $i > A.length$ and, since each loop iteration
    increases $i$ by 1, at that time we have $i = A.length + 1$ which assures
    (from the previous property) that $A[1, \dots, A.length]$ does not contain
    the element $\nu$.
\end{itemize}
\end{framed}

\item[2.1{-}4]{Consider the problem of adding two $n$-bit binary integers,
stored in two $n$-element arrays $A$ and $B$. The sum of the two integers should
be stored in binary form in an ($n + 1$)-element array $C$. State the problem
formally and write pseudocode for adding the two integers.}

\begin{framed}
The pseudocode is stated below. Integers are stored in little endian format.\\
\begin{algorithm}[H]
\SetAlgoNoEnd\DontPrintSemicolon
\BlankLine
\SetKwFunction{algo}{AddIntegers}
\SetKwProg{myalg}{}{}{}
\myalg{\algo{A, B}}{%
  \nl let $C[1, \dots, n + 1]$ be a new array\;
  \nl $C[1] = 0$\;
  \nl \For{$i = 1$ \KwTo $A.length$}{%
  \nl   $s = A[i] + B[i] + C[i]$\;
  \nl   $C[i] = s \mod 2$\;
  \nl   $C[i + 1] = s$ / $2$\; }
  \nl \Return{C} }
\end{algorithm}
\end{framed}

\end{enumerate}

\pagebreak

{\large Section 2.2 {--} Analyzing algorithms}

\begin{enumerate}

\item[2.2{-}1]{Express the function $n^3/1000 - 100n^2 - 100n + 3$ in terms of
$\Theta$-notation.}

\begin{framed}
  $\Theta(n^3)$.
\end{framed}

\item[2.2{-}2]{Consider sorting $n$ numbers stored in array $A$ by first finding
the smallest element of $A$ and exchanging it with the element in $A[1]$. Then
find the second smallest element of $A$, and exchange it with $A[2]$.  Continue
in this manner for the first $n - 1$ elements of $A$. Write pseudocode for this
algorithm, which is known as \textbf{\emph{selection sort}}. What loop invariant
does this algorithm maintain? Why does it need to run for only the first $n - 1$
elements, rather than for all $n$ elements? Give the best-case and worst-case
running times of selection sort in $\Theta$-notation.}

\begin{framed}
The pseudocode is stated below.\\
\begin{algorithm}[H]
\SetAlgoNoEnd\DontPrintSemicolon
\BlankLine
\SetKwFunction{algo}{SelectionSort}
\SetKwProg{myalg}{}{}{}
\myalg{\algo{A}}{%
  \nl \For{$i = 1$ \KwTo $A.length - 1$}{%
  \nl   $smallest = i$\;
  \nl   \For{$j = i + 1$ \KwTo $A.length$}{%
  \nl     \If{$A[j] < A[smallest]$}{%
  \nl         $smallest = j$\; } }
  \nl   $tmp = A[i]$\;
  \nl   $A[i] = A[smallest]$\;
  \nl   $A[smallest] = tmp$\; } }
\end{algorithm}

Here is the \emph{loop invariant}. At the start of each iteration of the
\textbf{for} loop of lines 1{--}8, the subarray $A[1, \dots, i - i]$ consists of
the $(i - 1)$ smallest elements of the array $A$ in sorted order.

\begin{itemize}
  \item \textbf{Initialization.} Before the \textbf{for} loop, $i = 1$ and $A[1,
    \dots, i - 1]$ constains no element.

  \item \textbf{Maintenance.} The body of the \textbf{for} loop looks on the
    subarray $A[i + 1, \dots, A.length]$ for a element that is smaller than
    $A[i]$. If a smaller element is found, their positions in $A$ are exchanged.
    Since the subarray $A[1, \dots, i - 1]$ already contains the $i$ smallest
    elements of $A$, the smaller element between $A[i]$ and $A[i + 1, \dots,
    A.length]$ is the i-th smallest element of $A$, which maintains our
    \emph{loop invariant} for the subarray $[1, \dots, i]$.

  \item \textbf{Termination.} The condition causing the \textbf{for} loop to
    terminate is that $i == A.length - 1$. At that time, $i = A.length = n$.
    Since (from the previous property) the subarray $A[1, \dots, n - 1]$
    consists of the $(n - 1)$ smaller elements $A$, the lasting element $A[n]$
    can only be the $n$-th smaller element.
\end{itemize}

It needs to run only for the first $(n - 1)$ element because, after that, the
subarray $A[1, \dots, n - 1]$ consists of the $(n - 1)$ smaller elements
of $A$ and the $n$-th element is already in the correct position.

Regardless of the content of the input array $A$, for
$i = 1, 2, \dots, (A.length - 1)$ the algorithm will always look for the $i$-th
element in the whole subarray $A = [i + 1, A.length]$. Thus, the algorithm takes
$\Theta(n^2)$ for every input.
\end{framed}

\item[2.2{-}3]{Consider linear search again (see Exercise 2.1-3). How many
elements of the input sequence need to be checked on the average, assuming
that the element being searched for is equally likely to be any element in
the array? How about in the worst case? What are the average-case and
worst-case running times of linear search in $\Theta$-notation? Justify your
answers.}

\begin{framed}
Lets consider an array of size $n$, where each element is taken from the set $1,
\dots, k$. If $k$ is not a function of $n$, its a constant. In the average case,
each comparison has probability $1/k$ to find the element that is being
searched, resulting in an average of $k$ comparisons. Thus, in the average case,
as a function of the input size, the algorithm takes $\Theta(k)$ = $\Theta(1)$.
The worst case occurs when $k >= n$, which takes $\Theta(n)$.
\end{framed}

\item[2.2{-}4]{How can we modify almost any algorithm to have a good best-case
  running time?}

\begin{framed}
Verify if the input is already solved. If it is solved, do nothing. Otherwise,
solve it with some algorithm.
\end{framed}

\end{enumerate}

\pagebreak

{\large Section 2.3 {--} Analyzing algorithms}

\begin{enumerate}

\item[2.3{-}1]{Using Figure 2.4 as a model, illustrate the operation of merge
sort on the array $A = \langle 3,41, 52, 26, 38, 57, 9, 49 \rangle$}

\begin{framed}

{\centering $ \displaystyle
\begin{aligned}
  \colorbox{gray!12}{\strut 3} \colorbox{gray!12}{\strut} \colorbox{gray!12}{\strut 9} \colorbox{gray!12}{\strut}
  \colorbox{gray!12}{\strut26} \colorbox{gray!12}{\strut} \colorbox{gray!12}{\strut38} \colorbox{gray!12}{\strut}
  \colorbox{gray!12}{\strut41} \colorbox{gray!12}{\strut} \colorbox{gray!12}{\strut49} \colorbox{gray!12}{\strut}
  \colorbox{gray!12}{\strut52} \colorbox{gray!12}{\strut} \colorbox{gray!12}{\strut57}\\
%
  \colorbox{gray!12}{\strut 3} \colorbox{gray!12}{\strut} \colorbox{gray!12}{\strut26} \colorbox{gray!12}{\strut}
  \colorbox{gray!12}{\strut41} \colorbox{gray!12}{\strut} \colorbox{gray!12}{\strut52} \colorbox{white}{\strut}
  \colorbox{gray!12}{\strut 9} \colorbox{gray!12}{\strut} \colorbox{gray!12}{\strut38} \colorbox{gray!12}{\strut}
  \colorbox{gray!12}{\strut49} \colorbox{gray!12}{\strut} \colorbox{gray!12}{\strut57}\\
%
  \colorbox{gray!12}{\strut 3} \colorbox{gray!12}{\strut} \colorbox{gray!12}{\strut41} \colorbox{white}{\strut}
  \colorbox{gray!12}{\strut26} \colorbox{gray!12}{\strut} \colorbox{gray!12}{\strut52} \colorbox{white}{\strut}
  \colorbox{gray!12}{\strut38} \colorbox{gray!12}{\strut} \colorbox{gray!12}{\strut57} \colorbox{white}{\strut}
  \colorbox{gray!12}{\strut 9} \colorbox{gray!12}{\strut} \colorbox{gray!12}{\strut49}\\
%
  \colorbox{gray!12}{\strut 3} \colorbox{white}{\strut}   \colorbox{gray!12}{\strut41} \colorbox{white}{\strut}
  \colorbox{gray!12}{\strut52} \colorbox{white}{\strut}   \colorbox{gray!12}{\strut26} \colorbox{white}{\strut}
  \colorbox{gray!12}{\strut38} \colorbox{white}{\strut}   \colorbox{gray!12}{\strut57} \colorbox{white}{\strut}
  \colorbox{gray!12}{\strut 9} \colorbox{white}{\strut}   \colorbox{gray!12}{\strut49}\\
\end{aligned} $ \par} % Necessary for centering to work

\end{framed}

\item[2.3{-}2]{Rewrite the \textsc{Merge} procedure so that it does not use
sentinels, instead stopping once either array $L$ or $R$ has had all its
elements copied back to $A$ and then copying the remainder of the other
array back into $A$.}

\begin{framed}
The pseudocode is stated below.\\
\begin{algorithm}[H]
\SetAlgoNoEnd\DontPrintSemicolon
\BlankLine
\SetKwFunction{algo}{Merge}
\SetKwProg{myalg}{}{}{}
\myalg{\algo{A, p, q, r}}{%
  \nl $n_1 = q - p + 1$\;
  \nl $n_2 = r - q$\;
  \nl let $L[1, \dots, n_1]$ and $R[1, \dots, n_2]$ be new arrays\;
  \nl \For{$i = 1$ \KwTo $n_1$}{%
  \nl   $L[i] = A[p + i - 1]$\; }
  \nl \For{$j = 1$ \KwTo $n_2$}{%
  \nl  $R[j] = A[q + j]$\; }
  \nl $i = 1$\;
  \nl $j = 1$\;
  \nl \For{$k = p$ \KwTo $r$}{%
  \nl  \If{$q + j > r$ \upshape{\textbf{or}} $L[i] \le R[j]$}{%
  \nl    $A[k] = L[i]$\;
  \nl    $i = i + 1$\; }
  \nl  \Else{%
  \nl    $A[k] = R[j]$\;
  \nl    $j = j + 1$\; } }
}{}
\end{algorithm}
\end{framed}

\item[2.3{-}3]{Use mathematical induction to show that when $n$ is an exact
power of 2, the solution of the recurrence
\begin{equation*}
  T(n) =
    \begin{cases}
      2 & \text{if}\ n=2 \text{,} \\
      2 T(n/2) + n & \text{if}\ n = 2^k \text{, for } k> 1
    \end{cases}
\end{equation*}
is $T(n) = n \lg n$.
}

\begin{framed}
  The base case is trivial, since $T(2) = 2 \lg 2 = 2$. To prove that it holds
  for $n > 2$ using mathematical induction, we need to show that if it holds for
  $n - 1$, it also holds for $n$. From the recurrence, $T(n) = 2T(n / 2) + n$.
  But by inductive hypothesis, $T(n / 2) = (n / 2) \lg (n / 2)$, so we get that:

  \begin{equation*}
  \begin{split}
    T(n) & = 2T(n/2) + n\\
         & = 2 (n / 2) \lg (n / 2) + n\\
         & = n \lg (n / 2) + n\\
         & = n (\lg (n) - \lg(2)) + n\\
         & = n \lg (n) - n + n\\
         & = n \lg (n).\\
  \end{split}
  \end{equation*}

\end{framed}

\item[2.3{-}4]{We can express insertion sort as a recursive procedure as
follows. In order to sort $A[1, \dots, n]$, we recursively sort $A[1, \dots,
n - 1]$ and then insert $A[n]$ into the sorted array $A[1, \dots, n - 1]$. Write
a recurrence for the worst-case running time of this recursive version of
insertion sort.}

\begin{framed}
The recurrence is stated below.
\begin{equation*}
  T(n) =
    \begin{cases}
      \Theta(1) & \text{if}\ n=1 \text{,} \\
      T(n - 1) + \Theta(n) & \text{if}\ n > 1\text{.}
    \end{cases}
\end{equation*}
It takes $\Theta(n^2)$.
\end{framed}

\item[2.3{-}5]{Referring back to the searching problem (see Exercise 2.1-3),
observe that if the sequence A is sorted, we can check the midpoint of the
sequence against $\nu$ and eliminate half of the sequence from further
consideration. The \textbf{\emph{binary search}} algorithm repeats this
procedure, halving the size of the remaining portion of the sequence each time.
Write pseudocode, either iterative or recursive, for binary search. Argue that
the worst-case running time of binary search is $\Theta(\lg n)$.}

\begin{framed}
The pseudocode is stated below.\\
\begin{algorithm}[H]
\SetAlgoNoEnd\DontPrintSemicolon
\BlankLine
\SetKwFunction{algo}{BinarySearch}
\SetKwProg{myalg}{}{}{}
\myalg{\algo{A, s, e, $\nu$}}{%
  \nl \If{$s > e$}{%
  \nl   \Return{\upshape{NIL}}\; }
  \nl $m = \floor{(s + e) / 2}$\;
  \nl \If{$\nu > A[m]$}{%
  \nl   \algo{A, $m + 1$, e, $\nu$}\; }
  \nl \ElseIf{$\nu > A[m]$}{%
  \nl   \algo{A, s, $m - 1$, $\nu$}\; }
  \nl \Else{%
  \nl   \Return{$m$}\; } }{}
\end{algorithm}

In each recursion level, the algorithm compares $\nu$ with the central element
$A[m]$. If $\nu = A[m]$, the element was found and it just returns the position.
If $A[m]$ is bigger (or smaller) than $\nu$, the algorithm discards the left
half (or the right half) of the array and continues recursively in the remaining
$\floor{(n - 1) / 2}$ elements. Each recursion element compares $\nu$ with
a single element of $A$, thus each level takes $\Theta(1)$. Since the number of
elements in the array is halved in each level, there will be at most $\lg n$
recursion levels. The algorithm then takes at most
$\lg n \times \Theta(1) = \Theta(\lg n)$.
\end{framed}

\item[2.3{-}6]{Observe that the while loop of lines 5{--}7 of the
\textsc{Insertion-Sort} procedure in Section 2.1 uses a linear search to scan
(backward) through the sorted subarray $A[1, \dots, j - 1]$. Can we use a binary
search (see Exercise 2.3-5) instead to improve the overall worst-case running
time of insertion sort to $\Theta(n \lg n)$?}

\begin{framed}
No, because even finding the correct position in $\lg n$, after each search the
algorithm will still need to shift up to $n$ the elements to keep the subarray
$A[1, \dots, j]$ sorted. The worst-case running time will remain $\Theta(n^2)$.
\end{framed}

\item[2.3{-}7]{($\star$) Describe a $\Theta(n \lg n)$-time algorithm that, given a set
$S$ of $n$ integers and another integer $x$, determines whether or not there
exist two elements in $S$ whose sum is exactly $x$.}

\begin{framed}
Start by sorting $S$ using \textsc{MergeSort}, which takes $\Theta(n \lg n)$.
For each element $i$ of $S$, $i = 1, \dots, n$, search the subarray $A[i + 1,
\dots, n]$ for the element $\nu = x - S[i]$ using \textsc{BinarySearch}. If
$\nu$ is found, return its position. Otherwise, continue for the next value of
$i$. It will perform at most $n$ searchs and each search takes $\Theta(\lg n)$.
The algorithm then takes $\Theta(n \lg n) + n \times \Theta(\lg n) = \Theta(n
\lg n)$.
\end{framed}

\end{enumerate}

\pagebreak

{\large Problems}

\begin{enumerate}

\item[2{-}1]{\textbf{\emph{Insertion sort on small arrays in merge sort}}\\
Although merge sort runs in $\Theta(n \lg n)$ worst-case time and insertion sort
runs in $\Theta(n^2)$ worst-case time, the constant factors in insertion sort can make
it faster in practice for small problem sizes on many machines. Thus, it makes
sense to \textbf{\emph{coarsen}} the leaves of the recursion by using insertion
sort within merge sort when subproblems become sufficiently small. Consider
a modification to merge sort in which $n/k$ sublists of length $k$ are sorted
using insertion sort and then merged using the standard merging mechanism, where
$k$ is a value to be determined.

\begin{enumerate}
\item[a.] Show that insertion sort can sort the $n/k$ sublists, each of length
  $k$, in $\Theta(n k)$ worst-case time.
\item[b.] Show how to merge the sublists in $\Theta(n \lg(n/k))$ worst-case time.
\item[c.] Given that the modified algorithm runs in $\Theta(n k + n \lg(n/k))$
  worst-case time, what is the largest value of $k$ as a function of $n$ for
  which the modified algorithm has the same running time as standard merge sort,
  in terms of $\Theta$-notation?
\item[d.] How should we choose $k$ in practice?
\end{enumerate}
}

\begin{framed}

\begin{enumerate}
\item[(a)] Sort $n/k$ sublists of length $k$ with insertion sort takes
$n/k \cdot \Theta(k^2) = \Theta(n/k \cdot k^2) = \Theta(nk)$.

\item[(b)] The naive solution is to extend the standard merging procedure to
merge n/k sublists at the same time, instead of two. Since there is $n/k$
sublists, in each iteration the algorithm takes $\Theta(n/k)$ to select the
lowest element among all the sublists. Since there are $n$ elements (thus $n$
iterations), the total complexity is $n \cdot \Theta(n/k) = \Theta({n^2}/k)$.

We can accomplish the requested $\Theta(n \lg(n/k))$ complexity by merging the
sublists pairwise, rather than merging them all at the same time. Lets first
consider the case in which the number of sublists is even. In the first level
there will be $n/(2k)$ pairs of sublists to merge and, since each sublist has
length $k$, each merge will take $\Theta(2k)$. Thus, the first level will take
$n/(2k) \cdot \Theta(2k) = \Theta(n)$. The next level will have half the number
of sublists and will take $n/(4k) \cdot \Theta(4k) = \Theta(n)$.  Since the
number of sublists is reduced by two on each level, the total number of levels
will be $\lg(n/k)$.  Thus, the total cost is
$\Theta(n) \cdot \lg(n/k) = \Theta(n \lg(n/k))$. When the number of sublists is
odd, it will need one additional level to merge the remaining sublist. Thus,
the total cost is $\Theta(n \lg(\ceil{n/k}))$.

\item[(c)] When $k = 1$ (smallest possible value for $k$), the modified
algorithm takes $\Theta(n \cdot 1 + n \lg (n/1)) = \Theta(n + n \lg n)$. When
$k$ grows, the first term grows and the second term decreases. Thus, since the
second term can not be greater than $n \lg n$, we just need to pay attention to
the first term. The algorithm then takes more than $\Theta(n \lg n)$ when
$nk > n \lg n \rightarrow k > \lg n$. Thus, the largest value of $k$ is $\lg n$.

\item[(d)] It depends of the constant factors of insertion sort and merge sort.
Since the cost of these constants may vary between different machines, in
practice one should choose the largest value of $k$ in which insertion sort
is faster then merge sort in a given machine.
\end{enumerate}

\end{framed}

\item[2{-}2]{\textbf{\emph{Correctness of bubblesort}}\\
Bubblesort is a popular, but inefficient, sorting algorithm. It works by
repeatedly swapping adjacent elements that are out of order.

\begin{algorithm}[H]
\SetAlgoNoEnd\DontPrintSemicolon%
\BlankLine
\SetKwFunction{algo}{BubbleSort}
\SetKwProg{myalgo}{}{}{}
\myalgo{\algo{A}}{%
 \nl \For{$i = 1$ \KwTo $A.length - 1$}{%
 \nl   \For{$j = A.length$ \upshape{\textbf{downto}} $i + 1$}{%
 \nl     \If{$A[j] < A[j - 1]$}{%
 \nl       exchange $A[j]$ with $A[j - 1]$\; } } } }{}
\end{algorithm}

\begin{enumerate}
\item[a.] Let $A'$ denote the output of \textsc{BubbleSort(A)}. To prove that
\textsc{BubbleSort} is correct, we need to prove that it terminates and that
\begin{equation*}
  A'[1] \le A'[2] \le \dots \le A'[n].
\end{equation*}
where $n = A.length$. In order to show that \textsc{BubbleSort} actually sorts,
what else do we need to prove?

\item[b.] State precisely a loop invariant for the \textbf{for} loop in lines
2{--}4, and prove that this loop invariant holds. Your proof should use the
structure of the loop invariant proof presented in this chapter.

\item[c.] Using the termination condition of the loop invariant proved in part
(b), state a loop invariant \textbf{for} the for loop in lines 1{--}4 that will
allow you to prove in- equality (2.3). Your proof should use the structure of
the loop invariant proof presented in this chapter.

\item[d.] What is the worst-case running time of bubblesort? How does it compare
  to the running time of insertion sort?
\end{enumerate}
}

\newpage

\begin{framed}

\begin{enumerate}
\item[(a)] $A'$ must be a permutation of $A$.

\item[(b)] Here is the \emph{loop invariant}. At the start of each iteration $j$
  of the for loop of lines 2{--}4, $A[j]$ is the smallest element of the subarray
$A[j, \dots, A.length]$.
\begin{itemize}
  \item \textbf{Initialization.} Prior to the first iteration of the loop, $j
    = n = A.length$, so the subarray $A[j, \dots, A.length]$ has only one
    element, and $A[j]$ is therefore the smallest element of the subarray $A[j,
    \dots, A.length]$.
  \item \textbf{Maintenance.} To see that each iteration maintains the loop
    invariant, let's suppose that $A[j - 1] > A[j]$. Because $A[j]$ is the
    smallest element of the subarray $A[j, \dots, A.length]$, after line
    4 exchanges the position of the elements $A[j]$ and $A[j - 1]$, $A[j - 1]$
    will be the smallest element of the subarray $A[j - 1, \dots, A.length]$.
    Incrementing $j$ (in the for loop update) reestablishes the loop invariant
    for the next iteration. If instead $A[j - 1] < A[j]$, nothing needs to be
    done and $A[j - 1]$ is already the smallest element of the subarray $A[j
    - 1, \dots, A.length]$.
  \item \textbf{Termination.} At termination, $j = i$. By the loop invariant
    $A[i]$ is the smallest element of the subarray $A[i, \dots, A.length]$.
\end{itemize}

\item[(c)] Here is the \emph{loop invariant}. At the start of each iteration $i$
of the for loop of lines 1{--}4, the subarray $A[1, \dots, i - 1]$ consists of
the $i$ smallest elements of $A$ in sorted order.
\begin{itemize}
  \item \textbf{Initialization.} Prior to the first iteration of the loop, we
    have $i = 1$, so that the subarray $A[1, \dots, i - 1]$ is empty. This empty
    subarray contains the $i - 1 = 0$ smallest elements of $A$ in sorted order.
  \item \textbf{Maintenance.} In each iteration $i$, the subarray $A[1, \dots,
    i - 1]$ constains the $i - 1$ smallest elements of $A$ in sorted order.
    After the for loop of lines 2{--}4, $A[i]$ will be the smallest element of
    the subarray $A[i, \dots, A.length]$ and thus the i-th smallest element of
    $A$. This implies that $A[1, \dots, i]$ will contain the $i$ smallest
    elements of $A$ in sorted order. Incrementing $i$ (in the for loop update)
    reestablishes the loop invariant for the next iteration.
  \item \textbf{Termination.} At termination, $i = A.length$. By the loop
    invariant the subarray $A[1, \dots, A.length - 1]$ consists of the smallest
    elements of $A$ in sorted order. Since $A[A.length]$ can only be the largest
    element of $A$, it is already in its correct position and the subarray $A[1,
    \dots, A.length]$ consists of the elements of $A$ in sorted order.
\end{itemize}

\item [(d)] The worst running time of \textsc{Bubble-Sort} is $\Theta(n^2)$,
which is the same of \textsc{Insertion-Sort}. However, the best running time of
\textsc{Insertion-Sort} is $\Theta(n)$ (when the array is already sorted) and
\textsc{Bubble-Sort} runs always in $\Theta(n^2)$.
\end{enumerate}

\end{framed}

\item[2{-}3]{\textbf{\emph{Correctness of Horner's rule}}\\
The following code fragment implements Horner's rule for evaluating a polynomial

\begin{equation*}
\begin{split}
  P(x) & = \sum_{k=0}^{n}{a_k x^k}\\
       & = a_0 + x(a_1 + x(a_2 + \cdots + x(a_{n - 1} + x a_n) \cdots)),
\end{split}
\end{equation*}
given the coefficients $a_0, a_1, \dots, a_n$ and a value for $x$:

\begin{algorithm}[H]
\SetAlgoNoEnd\DontPrintSemicolon%
\nl $y = 0$\;
\nl \For{i = n \upshape{\textbf{downto}} 0}{%
\nl $y = a_i + x \cdot y$\; }
\end{algorithm}

\begin{enumerate}
\item[a.] In terms of $\Theta$-notation, what is the running time of this code
fragment for Horner's rule?

\item[b.] Write pseudocode to implement the naive polynomial-evaluation
algorithm that computes each term of the polynomial from scratch. What is the
running time of this algorithm? How does it compare to Horner's rule?

\item[c.] Consider the following loop invariant:

At the start of each iteration of the \textbf{for} loop of lines 2{--}3,
\begin{equation*}
  y = \sum_{k = 0}^{n - (i + 1)}{a_{k + i + 1} x^k}.
\end{equation*}
Interpret a summation with no terms as equaling 0. Following the structure of
the loop invariant proof presented in this chapter, use this loop invariant
to show that, at termination, $y = \sum_{k = 0}^n{a_k x^k}$.

\item[d.] Conclude by arguing that the given code fragment correctly evaluates
a polynomial characterized by the coefficients $a_0, a_1, \dots, a_n$.
\end{enumerate}
}

\newpage

\begin{framed}
\begin{enumerate}
  \item[(a)] Since the body of the for loop of lines 2{--}3 consists of constant
    operations, the running time depends on the number of iterations of the
    loop. The running time is then $\sum_{i=0}^n 1 = n + 1 = \Theta(n)$.

\item[(b)] Here is the pseudocode of the \textsc{NaivePolynomialEvaluation}
algorithm:\\
\begin{algorithm}[H]
\SetAlgoNoEnd\DontPrintSemicolon%
 \nl $y = 0$\;
 \nl \For{$i = 0$ \KwTo $n$}{%
 \nl   $y_i = a_i$\;
 \nl   \For{$k = 1$ \KwTo $i$}{%
 \nl     $y_i = y_i x$\; }
 \nl   $y = y + y_i$\; }{}
\end{algorithm}

The running time of the above algorithm is
$\sum_{i=0}^n i = (n (n + 1))/2 = n^2/2 - n/2 = \Theta(n^2)$, which is slower
than the $\Theta(n)$ running time of Horner's rule.

\item[(c)] Here is the loop invariant proof:
\begin{itemize}
  \item \textbf{Initialization.} Prior to the first iteration of the for loop of
    lines 2{--}3, we have $y = 0$ and $i = n$. Replacing $i = n$ on the above
    loop invariant equation we have:
    \begin{equation*}
      y = \sum_{k = 0}^{n - n - 1}{a_{k + n + 1} x^k}
        = \sum_{k = 0}^{-1}{a_{k + n + 1} x^k}
        = 0,
    \end{equation*}
    which correctly corresponds to the initial value of $y$ on line 1.
  \item \textbf{Maintenance.} In each iteration $i$ of the loop, the previous
    value of $y$ is multiplied by $x$ and incremented by $a_i$ (line 3).
    Performing these two operations on the above loop invariant equation for an
    iteration $i$, we have:
    \begin{equation*}
      \begin{split}
        a_i + x \cdot \left(\sum_{k = 0}^{n - i - 1}{a_{k + i + 1} x^k}\right) =
        a_i + \left(\sum_{k = 0}^{n - i - 1}{a_{k + i + 1} x^{k+1}}\right) =
        a_i + \left(\sum_{k = 1}^{n - i}{a_{k + i} x^k}\right) =
        \left(\sum_{k = 0}^{n - i}{a_{k + i} x^k}\right),
      \end{split}
    \end{equation*}
    which correctly corresponds to the loop invariant equation in the iteration
    $i - 1$ (next iteration, after iteration $i$).
  \item \textbf{Termination.} At termination, we have $i = -1$, so that:
    \begin{equation*}
      y = \sum_{k = 0}^{n - (-1 + 1)}{a_{k + -1 + 1} x^k}
        = \sum_{k = 0}^n{a_k x^k}.
    \end{equation*}
\end{itemize}

\item[(d)] Since the loop invariant holds for all iterations and, at
termination, the loop invariant corresponds exactly to the polynomial
definition, we can assure that the code fragment correctly evaluates the
polynomial characterized by the coefficients $a_0, a_1, \dots, a_n$.
\end{enumerate}
\end{framed}

\item[2{-}4]{\textbf{\emph{Inversions}}\\
Let $A[1, \dots, n]$ be an array of $n$ distinct numbers. If $i < j$ and $A[i]
> A[j]$, then the pair $(i, j)$ is called an \textbf{\emph{inversion}} of A.

\begin{enumerate}
\item[a.] List the five inversions of the array
$\langle 2, 3, 8, 6, 1 \rangle$.

\item[b.] What array with elements from the set \{1, 2, \dots, n\}
has the most inversions? How many does it have?

\item[c.] What is the relationship between the running time of insertion sort
and the number of inversions in the input array? Justify your answer.

\item[d.] Give an algorithm that determines the number of inversions in any
permutation on $n$ elements in $\Theta(n \lg n)$ worst-case time.
(\emph{Hint: modify merge sort}.)
\end{enumerate}
}

\begin{framed}
\begin{enumerate}
  \item[(a)] (1, 4), (1, 5), (2, 5), (3, 5), (4, 5).
  \item[(b)] $\{n, n - 1, n - 2, \dots, 2, 1\}$.
    It has $\binom{n}{2} = n (n - 1) / 2$ inversions.
  \item[(c)] The number of operations of \textsc{Insertion-Sort} in an array
    $A$ is the same as the number of inversions in $A$.
  \item[(d)] The following pseudocode modifies \textsc{Merge-Sort} to count the
    number of inversions in $\Theta(n \lg n)$.\\
    \begin{algorithm}[H]
    \SetAlgoNoEnd\DontPrintSemicolon%
    \BlankLine
    \SetKwFunction{algo}{Inversions}
    \SetKwFunction{algomerge}{MergeInversions}
    \SetKwProg{myalgo}{}{}{}
    \myalgo{\algo{A, p, r}}{%
      \nl $inv = 0$\;
      \nl \If{$p < r$}{%
      \nl   $q = \floor{(p + r) / 2}$\;
      \nl   $inv = inv$ + \algo{A, p, q} + \algo{A, q + 1, r} + \algomerge{A, p, q, r}\;}
      \nl \Return{inv}
    }{}
    \end{algorithm}

    \begin{algorithm}[H]
    \SetAlgoNoEnd\DontPrintSemicolon
    \BlankLine
    \SetKwFunction{algo}{MergeInversions}
    \SetKwProg{myalg}{}{}{}
    \myalg{\algo{A, p, q, r}}{%
      \nl $inv = 0$\;
      \nl $n_1 = q - p + 1$\;
      \nl $n_2 = r - q$\;
      \nl let $L[1, \dots, n_1 + 1]$ and $R[1, \dots, n_2 + 1]$ be new arrays\;
      \nl \For{$i = 1$ \KwTo $n_1$}{%
      \nl   $L[i] = A[p + i - 1]$\; }
      \nl \For{$j = 1$ \KwTo $n_2$}{%
      \nl  $R[j] = A[q + j]$\; }
      \nl $L[n_1 + 1] = \infty$\;
      \nl $L[n_2 + 1] = \infty$\;
      \nl $i = 1$\;
      \nl $j = 1$\;
      \nl \For{$k = p$ \KwTo $r$}{%
        \nl  \If{$L[i] \le R[j]$}{%
      \nl    $i = i + 1$\; }
      \nl  \Else{%
        \nl    $inv = inv + (n_1 - i + 1)$\;
      \nl    $j = j + 1$\; } }
      \nl \Return{inv}\;
    }{}
    \end{algorithm}
\end{enumerate}
\end{framed}

\end{enumerate}

\end{document}
