\lhead{CLRS {--} Chapter 1 {--} The Role of Algorithms in Computing}

{\large Section 1.1 {--} Algorithms}

\begin{enumerate}

\item[1.1{-}1] Give a real-world example that requires sorting or
  a real-world example that requires computing a convex hull.

\begin{framed}
Sorting. In a dictionary, it is essential to use sorting so that one
can easily find the desired word.

Convex hull. After conducting a voting intention survey, it may be
interesting to know its coverage area. One can calculate the aproximate area by
projecting the covered cities to a two dimensional plane, obtain the convex hull
of the projected cities, and then compute the approximate area of the convex
hull.
\end{framed}

\item[1.1{-}2] Other than speed, what other measures of efficiency might
  one use in a real-world setting?

\begin{framed}
For algorithms in general, we can also optimize for low memory usage or low
power consumption. In machine learning algorithms, accuracy (hit rate) is also
considered a measure of efficiency.
\end{framed}

\item[1.1{-}3] Select a data structure that you have seen previously,
  and discuss its strengths and limitations.

\begin{framed}
Linked List is a basic data structure. Some of its strengths are:

\begin{itemize}
\item Given a pointer to an element in the list, we can insert an element after
  or before it in constant time.
\item Given a pointer to an element in the list, we can delete it in constant
  time.
\end{itemize}

Some of its limitations are:

\begin{itemize}
\item The pointers requires extra memory.
\item Since it only has pointers to the next element, it takes linear time to
  retrieve the i-th element.
\end{itemize}
\end{framed}

\item[1.1{-}4] How are the shortest-path and traveling-salesman problems
  given above similar? How are they different?

\begin{framed}
They are similar because both of them aims at minimizing the distance between
A and B, given a set of possible valid paths. However, the traveling-salesman
problem has an additional constraint: for a path to be valid, besides starting
in A and ending in B, it also needs to pass through a set of other points C, D,
\ldots, E before reaching B.
\end{framed}

\item[1.1{-}5] Come up with a real-world problem in which only the best
  solution will do. Then come up with one in which a solution that is
  ``approximately'' the best is good enough.

\begin{framed}
In a competition, each candidate received a score for her/his performance. To
obtain the ranking list of the candidates, only the best sorting solution is
accepted. Approximated sorting algorithms are not feasible in this situation.

Recently, Facebook computed the approximate degree of separation between every
two people in the world. Since Facebook has billion of users, it would take too
long to compute the solution that takes into account all the connections between
all the users. Also, an approximate result is very feasible in this case. They
then used an approximate to get the result of 3.57.
\end{framed}

\end{enumerate}

\pagebreak

{\large Section 1.2 {--} Algorithms as a technology}

\begin{enumerate}

\item[1.2{-}1] Give an example of an application that requires
  algorithmic content at the application level, and discuss the function of the
  algorithms involved.

\begin{framed}
The search engines we have today involves a lot of complex algorithms to work.
It needs a ranking algorithm to sort the search results appropriately. It is
also important to use a crawler that systematically browses and indexes the web
content. During searching, this indexed content is gathered and filtered from
the database using sophisticated algorithms.
\end{framed}

\item[1.2{-}2] Suppose we are comparing implementations of insertion
  sort and merge sort on the same machine. For inputs of size $n$, insertion sort
  runs in $8 n^2$ steps, while merge sort runs in $64 n \lg n$ steps. For which
  values of $n$ does insertion sort beat merge sort?

\begin{framed}
For input values less than or equal to 43, insertion sort beats merge sort. We
can ignore the case where $n = 1$, since a single element is already sorted by
definition.
\end{framed}

\item[1.2{-}3] What is the smallest value of $n$ such that an algorithm
  whose running time is $100 n^2$ runs faster than an algorithm whose running
  time is $2^n$ on the same machine?

\begin{framed}
The smallest value of $n$ is 15.
\end{framed}

\end{enumerate}

\pagebreak

{\large Problems}

\begin{enumerate}

\item[1{-}1] Comparison of running times. For each function $f(n)$ and
  time $t$ in the following table, determine the largest size $n$ of a problem
  that can be solved in time $t$, assuming that the algorithm to solve the problem
  takes $f(n)$ microseconds.

\begin{framed}
\centering
\footnotesize
\begin{tabular}{cccccccc}
  \toprule
  \multirow{2}{*}{$f(n)$} & \textbf{1} & \textbf{1} & \textbf{1} & \textbf{1} &
  \textbf{1} & \textbf{1} & \textbf{1}\\
  & \textbf{second} & \textbf{minute} & \textbf{hour} & \textbf{day} &
    \textbf{month} & \textbf{year} & \textbf{century}\\

  \midrule

  $\lg n$ & $2^{10^6}$ & $2^{10^7 \times 6}$ & $2^{10^8 \times 36}$ &
  $2^{10^8 \times 864}$ & $2^{10^9 \times 2592}$ &
  $2^{10^9 \times 31536}$ & $2^{10^{11} \times 31536}$ \\

  $\sqrt{n}$ & $10^{12}$ & $10^{14} \times 36$ & $10^{16} \times 1296$ &
  $10^{16} \times 746496$ & $10^{18} \times 6718264$ &
  $10^{18} \times 994519296$ & $10^{22} \times 994519296$ \\

  $n$ & $10^6$ & $10^7 \times 6$ & $10^8 \times 36$ &
  $10^8 \times 864$ & $10^9 \times 2592$ &
  $10^9 \times 31536$ & $10^{11} \times 31536$ \\

  $n \lg n$ & 62746 & 2801418 & 133378059 & 2755147513 & 71870856404 &
  797633893349 & 68610956750570 \\

  $n^2$ & $10^3$ & $7745$ & $10^4 \times 6$ & 293938 & 1609968 & 5615692 &
  561569229 \\

  $n^3$ & $10^2$ & 391 & 1532 & 4420 & 13736 & 31593 & 146645\\

  $2^n$ & 9 & 25 & 31 & 36 & 41 & 44 & 51\\

  $n!$ & 9 & 11 & 12 & 13 & 15 & 16 & 17\\

  \bottomrule
\end{tabular}
\end{framed}

\end{enumerate}
