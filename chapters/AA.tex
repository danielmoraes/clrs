\documentclass{report}

%                       MAIN PACKAGES                       %
% --------------------------------------------------------- %

\usepackage[english]{babel}
\usepackage{graphicx}
\usepackage{caption}
\usepackage{isolatin1}
\usepackage{amssymb}
\usepackage{textcomp}
\usepackage[usenames,dvipsnames]{color}
\usepackage{soul}
\usepackage{cancel}
\usepackage{booktabs}
\usepackage{multirow}
\usepackage{framed}
\usepackage{xcolor}
\usepackage{algpseudocode}
\usepackage{algorithm2e} % for psuedo code
\usepackage{courier}
\usepackage{mathtools} % loads amsmath
\usepackage{forest}

\usepackage{tikz}
\usepackage{tikz-qtree}
\usetikzlibrary{calc}

%                         MARGINS                           %
% --------------------------------------------------------- %

\hoffset=-0.5in
\voffset=-0.6in
\oddsidemargin=0pt
\topmargin=0pt
\headheight=12pt
\headsep=15pt
\textheight=690pt
\textwidth=543pt
\marginparsep=11pt
\marginparwidth=54pt
\footskip=25pt
\marginparpush=5pt
\paperwidth=597pt
\paperheight=845pt

%                        PAGE STYLE                         %
% --------------------------------------------------------- %

\usepackage{fancyhdr}
\pagestyle{fancy}
\lhead{CLRS {--} Appendix A {--} Summations}
\rhead{Daniel Bastos Moraes}
\cfoot{}
\rfoot{\thepage}
\renewcommand{\headrulewidth}{0.4pt}
\renewcommand{\footrulewidth}{0.4pt}
\newcommand{\Perp}{\perp\! \! \! \perp}

%                       PROPERTIES                          %
% --------------------------------------------------------- %

\setlength{\parindent}{0cm}

\makeatletter
\renewenvironment{framed}{%
 \def\FrameCommand##1{\hskip\@totalleftmargin
 \fboxsep=\FrameSep\fbox{##1}}%
 \MakeFramed {\advance\hsize-\width
   \@totalleftmargin\z@ \linewidth\hsize
   \@setminipage}}%
 {\par\unskip\endMakeFramed}
\makeatother

\makeatletter
\def\BState{\State\hskip-\ALG@thistlm}
\makeatother

\DeclarePairedDelimiter{\ceil}{\lceil}{\rceil}
\DeclarePairedDelimiter{\floor}{\lfloor}{\rfloor}
\DeclareMathOperator{\Exists}{\exists}
\DeclareMathOperator{\Forall}{\forall}

\def\figuredirectory{images}

\mathchardef\mhyphen="2D % Define a "math hyphen"

\DeclareMathOperator{\di}{d\!}
\newcommand*\Eval[3]{\left.#1\right\rvert_{#2}^{#3}}

%                        DOCUMENT                           %
% --------------------------------------------------------- %

\begin{document}

\small

{\large Section A.1 {--} Summation formulas and properties}

\begin{enumerate}

\item[A.1{-}1] {Find a simple formula for $\sum_{k = 1}^{n} (2k - 1)$.}

\begin{framed}
\begin{equation*}
\begin{aligned}
  \sum_{k = 1}^{n} (2k - 1) &= \sum_{k = 1}^{n} 2k - \sum_{k = 1}^{n} 1\\
                            &= 2 \sum_{k = 1}^{n} k - n\\
                            &= 2 \cdot \frac{1}{2} n (n + 2) - n\\
                            &= n^2 + n - n\\
                            &= n^2.
\end{aligned}
\end{equation*}
\end{framed}

\item[A.1{-}2] {($\star$) Show that
$\sum_{k = 1}^{n} 1/(2k - 1) = \ln(\sqrt n) + O(1)$ by manipulating the harmonic
series.}

\begin{framed}
\begin{equation*}
\begin{aligned}
  \sum_{k = 1}^{n} 1/(2k - 1) &= \frac{1}{1} + \frac{1}{3} + \frac{1}{5} + \dots + \frac{1}{2n - 3} + \frac{1}{2n - 1}\\
                              &= \left(\frac{1}{1} + \frac{1}{2} + \frac{1}{3} + \dots + \frac{1}{2n}\right)
                              - \frac{1}{2} \left(1 + \frac{1}{2} + \frac{1}{3} + \dots + \frac{1}{n}\right)\\
                              &= \sum_{k = 1}^{2n} \frac{1}{k} - \frac{1}{2} \sum_{k - 1}^{n} \frac{1}{k}\\
                              &= \ln{2n} + O(1) - \frac{1}{2} (\ln n + O(1))\\
                              &= \ln n + \ln 2 + O(1) - \frac{1}{2} \ln n - \frac{1}{2} O(1)\\
                              &= \frac{1}{2} \ln n + O(1)\\
                              &= \ln(\sqrt n) + O(1).
\end{aligned}
\end{equation*}
\end{framed}

\item[A.1{-}3] {Show that $\sum_{k = 0}^{\infty} k^2 x^k = x(1 + x)/(1 - x)^3$
for $0 < |x| < 1$.}

\begin{framed}
From Equation A.8, we have
\[
  \sum_{k = 0}^{\infty} k x^k = \frac{x}{(1 - x)^2}.
\]
differentiating both sides and multiplying by $x$, we have
\begin{equation*}
\begin{aligned}
  \sum_{k = 0}^{\infty} k^2 x^k &= x \cdot \frac{1 \cdot (1 - x)^2 - (2 \cdot (1 - x) \cdot (-1) \cdot x)}{(1 - x)^4}\\
                                &= x \cdot \frac{(1 - x)(1 - x) + (1 - x) \cdot 2x}{(1 - x)^4}\\
                                &= x \cdot \frac{(1 - x) + 2x}{(1 - x)^3}\\
                                &= \frac{x(1 + x)}{(1 - x)^3}.
\end{aligned}
\end{equation*}
\end{framed}

\newpage

\item[A.1{-}4] {($\star$) Show that $\sum_{k = 0}^{\infty} (k - 1)/2^k = 0$.}

\begin{framed}
\begin{equation*}
\begin{aligned}
  \sum_{k = 0}^{\infty} (k - 1)/2^k
  &= \sum_{k = 0}^{\infty} \left( \frac{k}{2^k} - \frac{1}{2^k} \right)\\
  &= \sum_{k = 0}^{\infty} k \frac{1}{2^k} - \sum_{k = 0}^{\infty} \frac{1}{2^k}\\
  &= \sum_{k = 0}^{\infty} k \left( \frac{1}{2} \right)^k - \sum_{k = 0}^{\infty} \left( \frac{1}{2} \right)^k\\
  &= \frac{(1/2)}{(1 - (1/2))^2} - \frac{1}{1 - (1/2)}\\
  &= \frac{(1/2)}{1 - 1 - (1/4)} - 2\\
  &= 4/2 - 2\\
  &= 0.
\end{aligned}
\end{equation*}
\end{framed}

\item[A.1{-}5] {($\star$) Evaluate the sum
$\sum_{k = 1}^{\infty} (2k + 1) x^{2k}$ for $|x| < 1$.}

\begin{framed}
\begin{equation*}
\begin{aligned}
  \sum_{k = 1}^{\infty} (2k + 1) x^{2k}
  &= \frac{d}{dx} \cdot \sum_{k = 1}^{\infty} x^{2k + 1}\\
  &= \frac{d}{dx} \cdot x \cdot \sum_{k = 1}^{\infty} x^{2k}\\
  &= \frac{d}{dx} \cdot x \cdot \sum_{k = 0}^{\infty} (x^2)^k - x\\
  &= \frac{d}{dx} \cdot x \cdot \frac{1}{1 - x^2} - x\\
  &= \frac{d}{dx} \cdot \frac{x - x (1 - x^2)}{1 - x^2}\\
  &= \frac{d}{dx} \cdot \frac{x^3}{1 - x^2}\\
  &= \frac{3 x^2 (1 - x^2) - (-2x) x^3}{(1 - x^2)^2}\\
  &= \frac{3 x^2 - 3 x^4 + 2 x^4}{(1 - x^2)^2}\\
  &= \frac{(3 - x^2) \cdot x^2}{(1 - x^2)^2}.
\end{aligned}
\end{equation*}
\end{framed}

\item[A.1{-}6] {Prove that
$\sum_{k=1}^{n} O(f_k(i)) = O(\sum_{k = 1}^{n} f_k(i))$ by using the linearity
property of summations.}

\begin{framed}
Skipped.
\end{framed}

\newpage

\item[A.1{-}7] {Evaluate the product $\prod_{k = 1}^{n} 2 \cdot 4^k$.}

\begin{framed}
We have
\[
  \prod_{k = 1}^n (2 \cdot 4^k) = 2^{\lg{\left(\prod_{k = 1}^n (2 \cdot 4^k)\right)}},
\]
and
\begin{equation*}
\begin{aligned}
  \lg{\left(\prod_{k = 1}^n (2 \cdot 4^k)\right)}
  &= \sum_{k = 1}^{n} \lg (2 \cdot 2^{2k})\\
  &= \sum_{k = 1}^{n} \lg 2^{2k + 1}\\
  &= \sum_{k = 1}^{n} (2k + 1)\\
  &= 2 \sum_{k = 1}^n k + \sum_{k = 1}^n 1\\
  &= n (n + 1) + n\\
  &= n (n + 2).
\end{aligned}
\end{equation*}

Thus,
\[
  \prod_{k = 1}^n (2 \cdot 4^k) = 2^{n (n + 2)}.
\]
\end{framed}

\item[A.1{-}8] {($\star$) Evalute the product $\prod_{k = 2}^{n} (1 - 1/k^2)$.}

\begin{framed}
We have
\[
  \prod_{k = 2}^{n} \left(1 - \frac{1}{k^2}\right) = 2^{\lg{\left( \sum_{k = 2}^{n} \lg{(1 - 1/k^2)} \right)}},
\]
and
\begin{equation*}
\begin{aligned}
  \sum_{k = 2}^{n} \lg{\left(1 - \frac{1}{k^2}\right)}
  &= \sum_{k = 2}^{n} \lg{\left( \frac{k^2 - 1}{k^2} \right)}\\
  &= \sum_{k = 2}^{n} \lg{\left( \frac{(k - 1)}{k} \cdot \frac{(k + 1)}{k} \right)}\\
  &= \sum_{k = 2}^{n} \left( \lg \left( \frac{k - 1}{k} \right) + \lg \left( \frac{k + 1}{k} \right) \right)\\
  &= \lg{\frac{1}{2}} + \lg{\frac{3}{2}} +
     \lg{\frac{2}{3}} + \lg{\frac{4}{3}} +
     \lg{\frac{3}{4}} + \lg{\frac{5}{4}} +
     \dots +
     \lg{\frac{n - 2}{n - 1}} + \lg{\frac{n}{n - 1}} +
     \lg{\frac{n - 1}{n}} + \lg{\frac{n + 1}{n}}\\
  &= \lg 1 - \lg 2 + \cancel{\lg 3} - \cancel{\lg 2} + \cancel{\lg 2} - \cancel{\lg 3}
           + \cancel{\lg 4} - \cancel{\lg 3} + \cancel{\lg 3} - \cancel{\lg 4} + \cancel{\lg 5} - \cancel{\lg 4} + \dots\\
  & \qquad + \cancel{\lg (n - 2)} - \cancel{\lg (n - 1)} + \cancel{\lg n} - \cancel{\lg (n - 1)}
           + \cancel{\lg (n - 1)} - \cancel{\lg n} + \lg (n + 1) - \lg n\\
  &= 0 - 1 + \lg (n + 1) - \lg n\\
  &= \lg (n + 1) - \lg (n) - 1.
\end{aligned}
\end{equation*}

Thus,
\[
  \prod_{k = 2}^{n} \left(1 - \frac{1}{k^2}\right) = 2^{(\lg (n + 1) - (\lg (n) + 1))}
  = \frac{2^{\lg{(n + 1)}}}{2^{\lg{(n) + 1}}}
  = \frac{n + 1}{2^{\lg n} \cdot 2}
  = \frac{n + 1}{2n}.
\]
\end{framed}

\end{enumerate}

\newpage

{\large Section A.2 {--} Bounding summations}

\begin{enumerate}

\item[A.2{-}1] {Show that $\sum_{k = 1}^{n} 1/k^2$ is bounded above by
a constant.}

\begin{framed}
\begin{equation*}
\begin{aligned}
  \sum_{k = 1}^{n} &=   1 + \sum_{k = 2}^{n} \frac{1}{k^2}\\
                   &\le 1 + \int_{1}^{n} \frac{dx}{x^2}\\
                   &=   1 + \left( \Eval{- \frac{1}{x}}{1}{n} \right)\\
                   &=   1 + \left( - \frac{1}{n} - \left( - \frac{1}{1} \right) \right)\\
                   &= 2 - \frac{1}{n}\\
                   &\le 2.
\end{aligned}
\end{equation*}
\end{framed}

\item[A.2{-}2] {Find an asymptotic upper bound on the summation
\[
  \sum_{k = 0}^{\floor{\lg n}} \ceil{n/2^k}.
\]
}

\begin{framed}
\begin{equation*}
\begin{aligned}
  \sum_{k = 0}^{\floor{\lg n}} \Bigl\lceil \frac{n}{2^k} \Bigr\rceil
  &=   n \cdot \sum_{k = 0}^{\floor{\lg n}}  \Bigl\lceil \frac{1}{2^k} \Bigr\rceil\\
  &\le n \cdot \sum_{k = 0}^{\lg n}  \left( \frac{1}{2^k} + 1 \right)\\
  &=   n \cdot \sum_{k = 0}^{\lg n}  \left( \frac{1}{2^k} \right) + \sum_{k = 0}^{\lg n} 1\\
  &=   n \cdot \frac{1}{1 - (1/2)} + \lg n + 1\\
  &=   2n + \lg n + 1\\
  &=   O(n).
\end{aligned}
\end{equation*}
\end{framed}

\item[A.2{-}3] {Show that the $n$th harmonic number is $\Omega(\lg n)$ by
splitting the summation.}

\begin{framed}
\begin{equation*}
\begin{aligned}
  \sum_{k = 1}^n \frac{1}{k} &\ge \sum_{i = 0}^{\floor{\lg n} - 1} \sum_{j = 0}^{2^i - 1} \frac{1}{2^i + j}\\
                             &\ge \sum_{i = 0}^{\floor{\lg n} - 1} \sum_{j = 0}^{2^i - 1} \frac{1}{2^{i + 1}}\\
                             &=   \sum_{i = 0}^{\floor{\lg n} - 1} \frac{1}{2} \cdot \sum_{j = 0}^{2^i - 1} \frac{1}{2^{i}}\\
                             &=   \sum_{i = 0}^{\floor{\lg n} - 1} \frac{1}{2}\\
                             &\ge \sum_{i = 0}^{\lg n - 2} \frac{1}{2}\\
                             &=   \frac{1}{2} (\lg (n) - 1)\\
                             &=   \Omega(\lg n).
\end{aligned}
\end{equation*}
\end{framed}

\newpage

\item[A.2{-}4] {Approximate $\sum_{k = 1}^{n} k^3$ with an integral.}

\begin{framed}
We have
\[
  \int_{0}^{n} x^3 dx \le \sum_{k = 1}^{n} k^3 \le \int_{1}^{n + 1} x^3 dx.
\]

For a lower bound, we obtain
\[
  \sum_{k = 1}^{n} k^3 \ge \int_{0}^{n} x^3 dx = \Eval{\frac{x^4}{4}}{0}{n} = \frac{n^4}{4} = \Omega(n^4).
\]

For the upper bound, we obtain
\[
  \sum_{k = 1}^{n} k^3 \le \int_{1}^{n + 1} x^3 dx = \Eval{\frac{x^4}{4}}{1}{n + 1} = \frac{(n + 1)^4 - 1}{4} = O(n^4).
\]

Thus,
\[
  \sum_{k = 1}^{n} k^3 = \Theta(n^4).
\]


\end{framed}

\item[A.2{-}5] {Why didn't we use the integral approximation (A.12) directly on
$\sum_{k = 1}^{n} 1/k$ to obtain an upper bound on the $n$th harmonic number?}

\begin{framed}
Applying (A.12) directly, we obtain
\[
  \sum_{k = 1}^{n} \frac{1}{k} \le \int_{0}^{n} \frac{1}{x} dx,
\]
but the function $1/x$ is undefined for $x = 0$ (because of the division by
zero).
\end{framed}

\end{enumerate}

\newpage

{\large Problems}

\begin{enumerate}

\item[A{-}1]{\textbf{\emph{Bounding summations}}\\
Give asymptotically tight bounds on the following summations. Assume that
$r \ge 0$ and $s \ge 0$ are constants.
\begin{enumerate}
  \item[a.] $\sum_{k = 1}^{n} k^r$.
  \item[b.] $\sum_{k = 1}^{n} \lg^s k$.
  \item[c.] $\sum_{k = 1}^{n} k^r \lg^s k$.
\end{enumerate}
}

\begin{framed}
\begin{enumerate}
\item[(a)] For a lower bound, we have
\begin{equation*}
\begin{aligned}
  \sum_{k = 1}^{n} k^r &\ge \int_{0}^{n} x^r dx\\
                       &=   \Eval{\frac{x^{(r + 1)}}{r + 1}}{0}{n}\\
                       &=   \frac{n^{(r + 1)}}{r + 1} - \frac{0^{(r + 1)}}{r + 1}\\
                       &\ge n^{(r + 1)}\\
                       &=   \Omega(n^{(r + 1)}),
\end{aligned}
\end{equation*}
and for the upper bound, we have
\[
  \sum_{k = 1}^{n} k^r \le \sum_{k = 1}^{n} n^r = n^{(r + 1)} = O(n^{(r + 1)}).
\]

Thus,
\[
  \sum_{k = 1}^{n} = \Theta(n^{(r + 1)}).
\]

\item[(b)] For a lower bound, we have
\begin{equation*}
\begin{aligned}
  \sum_{k = 1}^{n} \lg^{s} k &=   \sum_{k = 1}^{n/2} \lg^{s} k + \sum_{k = n/2 + 1}^{n} \lg^{s} k\\
                             &\ge \sum_{k = 1}^{n/2} 0 + \sum_{k = n/2 + 1}^{n} \lg^{s}{\left(\frac{n}{2}\right)}\\
                             &=   \frac{n}{2} \lg^{s}{\left(\frac{n}{2}\right)}\\
                             &=   \frac{n}{2} \lg^{s}{n} - \frac{n}{2} \lg^{s}{2}\\
                             &\ge \frac{1}{2} n \lg^{s} n - \frac{1}{2} n\\
                             &=   \Omega(n \lg^s n),
\end{aligned}
\end{equation*}
and for the upper bound, we have
\[
  \sum_{k = 1}^{n} \lg^{s} k \le \sum_{k = 1}^{n} \lg^{s} n = n \lg^s n = O(n \lg^s n).
\]
Thus,
\[
  \sum_{k = 1}^{n} \lg^{s} k = \Theta(n \lg^s n).
\]

\item[(c)] It is easy to see that this summation is greater than the one from
item (a). Thus, it is $\Omega(n^{(r + 1)})$. Also, we have
\[
  \sum_{k = 1}^{n} k^r \lg^{s} k \le \sum_{k = 1}^{n} n^r \lg^{s} n = O(n^{(r + 1)} \lg^s n).
\]

Thus, I guess it is $\Theta(n^{(r + 1)} \lg^s n)$.

\end{enumerate}
\end{framed}

\end{enumerate}

\end{document}
